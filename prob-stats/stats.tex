%%=====================================================================================
%%
%%       Filename:  stats.tex
%%
%%    Description:  Notes for statistics written up nicely
%%
%%        Version:  1.0
%%        Created:  05/01/19
%%
%%         Author:  Josh Felmeden (), nk18044@bristol.ac.uk
%%
%%=====================================================================================

% Preamble
\documentclass[11pt,a4paper,titlepage]{scrartcl}
\usepackage[english]{babel}
\typearea{12}

% Set indentation and line skip for paragraph
\setlength{\parindent}{0em}
\setlength{\parskip}{1em}
\usepackage[margin=2cm]{geometry}
\addtolength{\textheight}{-1in}
\setlength{\headsep}{.5in}

% Math
\usepackage[]{amsmath} 

% Headers setup
\usepackage{fancyhdr}
\pagestyle{fancy}
\lhead{Statistics: Nicely written up}
\rhead{Josh Felmeden}

% Title
\author{Josh Felmeden}
\date{January \\ 2019}
\title{Statistics: Nicely written up}

% Links
\usepackage{hyperref} 

% Start document
\begin{document}
\maketitle
\tableofcontents
\newpage

\section{Regression}%
\label{sec:regression}
When we look at statistics, there are a lot of things that we need to
consider. One of these things is to consider how well something fits the
line, or the expected results of the data etc. If we take the example of
Fitt's law, which is the time required to \textbf{acquire a target} of
size \textit{w} at a distance of \textit{d}, and we describe this as $T =
a + b \log(1+ \frac{d}{w})$. Say we were trying to throw a paper ball into
the bin. It would get harder if the bin were smaller or further away, and
this fits with the Fitt's law. But, where does the equation come from?
Well, when we plot it, we get a kinda straight line. 

\textbf{Regression} is a technique for determining the statistical
relationship between two or more variables where a change in some
\textit{dependent variable} is associated with, and depends on a change,
in one or more \textit{independent variables}. This is the most basic
technique for machine learning.

There are two basic kinds of regression: linear, and \textit{multiple}
linear regression. There are some non-linear regression methods, but we
don't need to worry about that bad boy right now. Regression is pretty
useful because it helps financial and professional investment. It can also
help to predict sales for a company based on weather, or previous sales
etcetera.

So, the formula for linear regression is:
\begin{equation*}
    Y = a + bX (+ u)
\end{equation*}

Multiple regression is:
\begin{equation*}
    Y = a + b_1 X_1 + b_2 X_2 + b_3 X_3 + \cdots + b_t X_t + u
\end{equation*}

But we don't really need to worry about the multiple regression.

In this, $Y$ is the variable that we're trying to predict (or the
\textbf{dependent} variable), while $X$ is the variable that we're using
to predict $Y$ (or the \textbf{independent} variable). $a$ is the
intercept, $b$ is the slope and $u$ is the regression residual.

Regression takes a group of random variables that we think will predict
$Y$, and it attempts to find a mathematical relationship between them.
This relationship is usually in the form of a straight line (known as
linear regression) that best approximates all the individual data points.

The residual is also known as the deviation, and we don't really worry
about it in this module.

But how can you be sure that a line we draw is a good fit for the data?
You can actually compute the goodness of fit with a number of methods,
like the \textit{standard error of the estimate} or \textit{R squared}.

\subsection{Standard error of estimate}%
\label{sub:standard-error}
The equation for this is:
\begin{equation*}
    \sqrt{\frac{\sum {(\vdots)}^2}{(\text{sample size} - 2)}}
\end{equation*}

The bottom part of the fraction is also known as the \textit{degree of
freedom}. The result of this gives a standard error in the metric of the
data. The lower it is, the better it is.

\subsection{R squared}%
\label{sub:r-squared}
The equation for this one is:
\begin{equation*}
    \frac{\sum{{(\vdots)}^2}}{\sum{{(\vdots)}^2}}
\end{equation*}

\begin{equation*}
    \frac{\sum{{(\text{estimated } \hat y - \text{mean }
    y)}^2}}{\sum{(\text{actual } y - \text{mean } y)}^2}
\end{equation*}
You could also find it in a format which is \textit{more generic}. R
squared gives a \textbf{percentage} result, and 100\% means that it is a
perfect fit (however, we say that >70\% is acceptable).

Say we have graph that has the following values for the standard error and
R squared:
\begin{equation*}
    T = 2.3 + 1.1 ID \\
    R^2 = 0.97
\end{equation*}

How can we be sure that it's a good fit for every human (say that it is a
test modelling some human performance). Well, we can gain additional
confidence by \textbf{repeating}. We gain trust in a model if it fits the
data with \textit{little error} when it:
\begin{enumerate}
    \item is verified with \textit{a lot of data}.
    \item holds across \textit{very different people}.
    \item is verified in \textit{independent studies}.
\end{enumerate}

\subsection{What can it be used for?}%
\label{sub:usage}
Linear regressions can be used for predicting things, like Ebay's online
auction prices using functional data analysis, or the number of passersby
who will pass in front of a public ad and use the data for choosing
advertisement prices. It's really quite useful, to be fair.

\newpage

\section{Comparing things and hypothesis testing}%
\label{sec:comparing}
We're moving onto a \textit{discrete} independent variable, with a
\textit{continuous} dependent variable. It's normally distributed, which
means that we have three choices for data comparison:
\begin{enumerate}
    \item 2 groups
    \begin{enumerate}
        \item[a)] Non paired = T-Test
        \item[b)] Paired = Paired T-Test
    \end{enumerate}
    \item $>2$ groups = ANOVA
\end{enumerate}

We're gonna start with a kind of weird example Say we have some magic
shoes, and you run the 10 metres with it a lot. You also (\textit{wierdly})
have a log of the times you've run with it. One day,
you come home, and someone has moved your shoes. You think that someone
may have taken your gorgeous shoes, and replaced it with some
\textbf{identical looking ones}. GASP! You look at the shoes, and they do
indeed look the same, but you're still concerned. What do you do?

If you answered ``run the 10 metres with the weird shoes a lot of times",
you'd be \textbf{correct!}. You do so, and this is what you see that your
times have changed. You're more consistently getting slower times, so is
this proof that some thieving bastard has filched your shoes?

Actually, it's kind of impossible to know. It's a common limitation of
science, no matter how many times something happens, it could
\textit{always} be chance. The good news is that the more samples you
take, the more your confidence increases, so you can be
\textbf{arbitrarily sure}.

Now, you get your shoes returned to you, but a week later, the same thing
happens again. So, you run the 10-metre race a few times, and the
distribution matches your log files almost perfectly. Are these your
shoes? YOU STILL DON'T KNOW. That's right, you don't know, because it
could be a really good copy of your shoes. The odds of this being a
different pair of shoes, though, can't be computed. Why not? Because there
are always two explanations:
\begin{enumerate}
    \item same shoes
    \item different shoes
\end{enumerate}

What is the point of this stupid module? Well, if we use stats, it turns
out that we can be kind of sure. \textbf{Statistical significance} is a
result that is unlikely to have occurred by chance. A \textit{t-test}
returns a p-value. A p-value is such that if it is lower than the
significance level, then the results are colloquially known as
`statistically significant'. The usual level that we go for is 5\% (or a
value of 0.05).

\subsection{Null Hypothesis}%
\label{sub:Null Hypothesis}
The definition of the null hypothesis is \textbf{both sets of data are
from the same mechanism}. We are running tests to try to \textit{reject}
the null hypothesis. If, when we compare the two groups, there is no
statistically significant difference between the two, it doesn't mean that
there is not difference in reality. It just means that there's not enough
evidence to reject the null hypothesis. In other words, it \textit{fails
to reject the null hypothesis.}

Okay, new example time: let's say that we're making a new input device. It
can't be better than a mouse, but you want to prove that it's \textit{as
good as the mouse.} How do you go about that?

What if you run a test, and if the stats come out as insignificant, you
write that `the tests showed that there was \textit{no difference'?} Well,
you'd be stupid and incorrect because no significant difference means
absolutely zilch. So how in the Christ do we prove that the two mechanisms
are the same? (\textit{lotta unanswered questions here boss}.)

You can't (\textit{shock}). The only thing that you can write is `our test
did not find a significant difference'. Boring, right?

Say you want to test the effect of two soporific drugs on the amount of
sleep. You take 10 participants and make them sleep to get their normal
sleep time (this is the control variable). You then give them drug 1
and note the difference of sleep time. You do the same for drug 2. You end
up with the following results:
\begin{center}
    \begin{tabular}{c|c}
        \textbf{Sleep extra drug 1} & \textbf{Sleep extra drug 2} \\
        \hline
        1. 0.7 & 1. 1.9 \\
        \hline
        2. -1.6 & 2. 0.8 \\
        \hline
        3. -0.2 & 3. 1.1 \\
        \hline
        4. -1.2 & 0.1 \\
        \hline
        5. -0.1 & -0.1 \\
        \hline
        6. 3.4 & 6. 4.4 \\
        \hline
        7. 3.7 &  7. 5.5 \\
        \hline
        8. 0.8 & 8. 1.6 \\ \hline
        9. 0.0 & 9. 4.6 \\ \hline
        10. 2.0 & 10. 3.4 
    \end{tabular}
\end{center}

Because you want to test the effects of the drug, and \textit{all
participants did both conditions}, then the data is paired. If the
subjects had not done both drugs (so take new participants for drug 2),
then the data is unpaired.

In this experiment (the between subject one), if you crunch the numbers,
you'll find that we can't
reject the null hypothesis, since the p-value equates to 0.07939. What
you'd write for this is ``An unpaired student t-test showed no significant
difference between the two drugs".

In the within subject experiment, you'll find that the p-value ends up
being 0.002833, so you'd write ``A paired t-test showed significant
difference between the two drugs (two-tailed t(9) = -4.0621, $p<0.05$)"

Oh, try to design your studies within-subject, because it increases the
chance you find a smaller p-value. If you don't you're gonna need twice as
many participants.

\subsection{One tail vs two tail}%
\label{sub:tail}
A \textbf{two tailed} t-test is asking whether the effect of one drug is
greater than or less than the effect of the other, while \textbf{one
tailed} is only one side of the effect (like is drug 1 better than drug 2
OR is drug 1 less than drug 2). You'll normally use two-tails, but if you
can use a one-tail, then go ahead because it will increase the chance to
reach a smaller p-value.

\subsection{Multiple variables}%
\label{sub:multiple}
What do you do if you have more than two variables? Going back to the
input devices example again, say we're comparing a mouse to a track pad and
a stylus. How do you work out which one is better? What you could do is to
t-test each one against the others, but you then have a lot of errors
adding up and it gets sad. There are two solutions to this:

\subsubsection{Bonferroni correction}%
\label{ssub:bonferroni}
When testing $n$ hypotheses, test each one against 0.05/n. In the example
above, we should use $0.05 \div 3$ as a significant threshold instead of
0.05.

\subsubsection{ANOVA}%
\label{ssub:anova}
This is the analysis of variance to compare multiple variables. There are
two kinds of anova:
\begin{enumerate}
    \item \textbf{one-way anova} -- one variable, multiple levels.
    \item \textbf{two-way anova} -- two variables, multiple levels.
\end{enumerate}

We're going to look at anova over the next few sections.

\section{Designing an experiment}%
\label{sec:designing}
Say we do an experiment, and we get some results. We put the results into
a table (excel is pretty good for that), and then save it as a `.csv'
file (a comma separated virgule). We can then do some cool things with R.
I'm not going to demonstrate them here, because I don't think we need it
for the exam, but I'll relay the basics.

There's quite a simple flowchart to follow when we're designing our own
experiment:
\begin{enumerate}
    \item[1)] Develop the research question or hypothesis
    \item[2)] What are the variables going to be?
    \item[3 a)] Will it be within or between subjects?
    \item[3 b)] Do you need any counter balancing?
    \item[4)] how many repetitions do you need?
    \item[5)] Look at the raw data
    \item[6)] Look at the distributions
    \item[7 a)] Check for normality
    \item[7 b)] Run some stats
    \item[8)] Conclude
\end{enumerate}

When developing the research question, you need to identify a statement
that identifies some phenomenon that you want to study. An example of this
is `in our experiment, I believe that rewards will improve memorisation
skills'. The hypothesis is a provisional answer to a research question, so
in our experiment, we'd have `group chocolate will have a higher
memorisation score than the group that doesn't have a reward'.

The dependent variable is the event studied and expected to change
whenever the independent variable is altered. In this experiment, the
independent variable is the group type (which one gets chocolate), and the
dependent variable is the memorisation score. Everything else is a control
variable.

The \textbf{confounding variable} is the extraneous variables that
\textit{correlates} with both the dependent variable and the independent
variable. For example, saying that ice cream consumption leads to murder
has a confounding variable of the weather. Apparently you're more likely
to be murdered if it's hot out. Who knew? We're not out to prove
correlation, we're out to show causality (so some A causes some B).

Do we have any confounding variables? Yeah, because our experiment isn't
greatly designed. There's gender, age, background, what you ate before,
etcetera, etcetera. But what can we do about it? Well, we can avoid them
by controlling as much as you can about the environment. If not, make it
an independent variable. Some of the things are inherent noise (like basic
human individuality), so you can use more participants to get statistical
power.

The goal of a quantitative study is to find a signal in a lot of noise.

In experimental design, you aim to maximise your chances of finding the
signal and not the noise.

One of the main things that you need to avoid is \textbf{systematic
biases} such as learning effect or fatigue. These will give you false
results. The other thing you need to avoid is \textbf{random noise}. It
makes your results non-significant. Clever experimental design is all
about keeping noise down.

\subsection{Within vs Between}%
\label{sub:within-between}

So, what's the deal with within vs between? Well, within means that all
participants are doing the same thing. Between means that participants do
only some of the conditions. For example, between would be group 1 vs
group 2, while within would be all participants do both group 1 and group
2.

Within subject experiments suffer less user variation and the statistical
power with less participants, while between means that there are no biases
from other conditions (like the transfer of learning or some participants
getting used to the experiment from the other group).

In the experiment with the chocolate, it was between because of the
rewards. Half of the participants did the control condition while the
other half had the reward condition.

Imagine a within subject experiment where we test how fast we click an
icon. The participants do all the conditions: the start with the trackpad
and finish with the mouse. Is this a good idea? No because they have
something known as the learning effect, where they learn the thing they
need to do and so will perform better in the second experiment. This is
where the next part comes in.

\subsubsection{counterbalancing}%
\label{ssub:counterbalancing}
This is a method of avoiding confounding among variables and involves
presenting the conditions in a different order. One approach is to use a
Latin square. What is this? Well, it's an $n \times n$ array filled with
$n$ different Latin letters, each occurring exactly once in each column.
Basically, you make some people do the first group, followed by the second
group, and then some others do the second group followed by the first
group. This gives a much more accurate picture of the results.

\subsection{How many trials?}%
\label{sub:trials}
Ideally, you want as many trials as you can, but try to keep the
experiment to around 30 or 40 minutes. In the experiment with the
chocolate, we only did one because of time, but you should do more to
reduce noise. 

\subsection{ANOVA in use}%
\label{sec:anova-usage}

Let's say that we want to add a third imaginary group to the experiment
with the chocolate. Let's say that if they had the smallest memorisation
score, then they get a slap on the wrist (obviously not in real life, only
in hypothetical terms). 

How can we compare them? Can we use t-tests? Yeah, but we gotta use
Bonferoni correction. Remember, that it's two tailed and it is unpaired
data.

Another test that we can do is the ANOVA test. We have 3 different
conditions (or 1 factor with 3 different levels, so we can do a one-way
ANOVA. ANOVA can be of two kinds (if you remember from above). If we do
the tests on some data, we can write: `A one-way ANOVA showed a
significant effect on time for the variable group (F2.57 = 154.88, $p <
0.05$)' and then: `Post-hoc comparison t-tests (using Bonferoni correction)
showed significant difference between the group C and the group A
($p<0.05$)'. You could even give some mean values if you really wanted to
flex.
\end{document}
