%%=====================================================================================
%%
%%       Filename:  comp-arch.tex
%%
%%    Description:  A comprehensive list of notes from teaching block 1, year 1.
%%
%%        Version:  1.0
%%        Created:  16/12/18
%%
%%         Author:  Josh Felmeden (), nk18044@bristol.ac.uk
%%      Copyright:  Copyright (c) 2018, Josh Felmeden
%%
%%
%%=====================================================================================

% Preamble {{{
\documentclass[11pt,a4paper,titlepage,dvipsnames,cmyk]{scrartcl}
\usepackage[english]{babel}
\typearea{12}
% }}}

% Set indentation and line skip for paragraph {{{
\setlength{\parindent}{0em}
\setlength{\parskip}{1em}
\usepackage[margin=2cm]{geometry}
\addtolength{\textheight}{-1in}
\setlength{\headsep}{.5in}
% }}}

% Headers setup {{{
\usepackage{fancyhdr}
\pagestyle{fancy}
\lhead{Introduction to Computer Architecture: The (almost) Comprehensive
Notes}
\rhead{Josh Felmeden}
% }}}

%Listings 
% Listings {{{
\usepackage[]{listings,xcolor} 
\lstset
{
    breaklines=true,
    tabsize=3,
    showstringspaces=false
}

\definecolor{lstgrey}{rgb}{0.05,0.05,0.05}
\makeatletter
\lstset{language=[Visual]Basic,
    backgroundcolor=\color{lstgrey},
    frame=single,
    xleftmargin=0.7cm,
    frame=tlbr, framesep=0.2cm, framerule=0pt,
    basicstyle=\lst@ifdisplaystyle\color{white}\footnotesize\ttfamily\else\color{black}\ttfamily\fi,
    captionpos=b,
    tabsize=2,
    keywordstyle=\color{Magenta}\bfseries,
    identifierstyle=\color{Cyan},
    stringstyle=\color{Yellow},
    commentstyle=\color{Gray}\itshape
}
\makeatother
\renewcommand{\familydefault}{\sfdefault}

% }}}

% Other packages {{{
\usepackage{bookmark}
\usepackage{hyperref}
\usepackage{hhline}
\usepackage{lmodern}
\usepackage[]{amsmath} 
\usepackage[]{booktabs} 
\usepackage{tcolorbox} 
\usepackage[symbol]{footmisc} 
\renewcommand{\thefootnote}{\fnsymbol{footnote}}
\renewcommand{\familydefault}{\rmdefault}
% }}}

% Title {{{
\title{Introduction to Computer Architecture: The (almost) Comprehensive
Notes}
\author{Josh Felmeden}
% }}}

% Start document
\begin{document}
\pagenumbering{roman}
\maketitle

\tableofcontents
\newpage
\pagenumbering{arabic}

\part{Maths and Physical Parts}
\section{Integer representation and arithmetic}%
\label{sec:reallyboring}
First things first, we need to look at the ways that numbers are added together
(I chose to skip over how numbers are stored in the computer because I think
this is boring and if you need help with this then you really are beyond all
hope.). Ultimately, we've started with addition to start with, so we're going to
look at simple circuits for addition in a computer. By the way, if we have a
letter with a hat on (namely: $\hat x$), this means that it's a bit sequence
representing some integer $x$. This leaves us with this:
\begin{align*}
    \hat x \mapsto x \\
    \hat y \mapsto y \\
    \hat r \mapsto r
\end{align*}

Alongside this, we have the following relationship:
\begin{equation}
    r = x + y
\end{equation}

The question is, how do we take this and represent it using boolean algebra?
Well, we're going to try:
\begin{equation}
    \hat r = F(\hat x, \hat y)
\end{equation}

Where $F$ is some boolean expression. What this means is that the $+$ operator
has a similar result than $F$. What the bloody hell is this function? Let's have
a look.

It's actually not that bad. If we look at how humans do addition, we have:
\begin{align*}
    &\hat x = 1 \quad 0 \quad 7 \\
    + \quad &\hat y = 0 \quad 1 \quad 4 \\
        &\rule{60pt}{1pt} \\
           &c = 0 \quad 1 \quad 0 \\
           &\hat r = 1 \quad 2 \quad 1
\end{align*}
Where c is the carry (we also have 0 as a carry out here). The same thing can be
represented in binary:
\begin{align*}
    107_{10} = \quad &\hat x = 0 \quad 1 \quad 1 \quad 0 \quad 1 \quad 0 \quad 1 \quad
    1 \\
    + 14_{10} = \quad &\hat y = 0 \quad 0 \quad 0 \quad 0 \quad 1 \quad 1 \quad 1
    \quad 0 \\
      &\rule{150pt}{1pt} \\
      &c = 0 \quad 0 \quad 0 \quad 1 \quad 1 \quad 1 \quad 0 \quad 0 \\
      &\hat r = 0 \quad 1 \quad 1 \quad 1 \quad 1 \quad 0 \quad 0 \quad 1
\end{align*}

Now, we're going to create a really simple algorithm (it's going to be called
ADD), and is going to look like $\text{ADD}(\hat x, \hat y, n, b, ci)$, where b
is the base, ci is the carry in and n is the length of x and y. We would then
have the algorithm as follows:
\begin{lstlisting}[]
for i = 0 to (n-1)
    r(i) += (x(i) + y(i) + c(i)) mod b
    if(x(i) + y(i) + c(i) < b)
        c(i+1) = 0
    else
        c(i+1) = 1
    end if
next
co = c(n)
return r, co
\end{lstlisting}

Let's step through this algorithm:
\begin{gather*}
    \hat x = \langle 7, 0, 1 \rangle \mapsto 107_{10} \\
    \hat y = \langle 4, 1, 0 \rangle \mapsto 14_{10} \\
    n = 3, \ b = 10, \ ci = 0, \\
    \text{ADD}(\hat x, \hat, y, 3, 10, 0)
\end{gather*}

\begin{center}
    \begin{tabular}{|c|c|c|c|}
        \hline
        i & $\hat x_i, \ \hat y_i, \ c_i$ & $\hat x_i + \hat y_i + c_i$ &
        $c_{i+1}, \hat r_i$\\
        \hline
        0 & 7, 4, 0 & 11 & 1, 1 \\
        \hline
        1 & 0, 1, 1 & 2 & 0, 2 \\
        \hline
        2 & 1, 0, 0 & 1 & 0, 1 \\
        \hline
    \end{tabular}
\end{center}

Where the at the end, $\hat r = \langle 1 2 1 \rangle$, as stated by the last
column.

In the algorithm above, the bit inside the for loop can be represented by
$F_i$, where it has the inputs $\hat x_i, \hat y_i, \hat c_i$ and has
outputs $\hat r_i,c_{xi+1}$. We don't have to know what the function is,
but we can write down their behaviour. Because we know what should
happen\footnote[1]{These tables took so damn long please appreciate them}.

\begin{center}
    \begin{tabular}{|c c c|c|c|}
        \hline
        $c_i$ & $\hat x_i$ & $\hat y_i$ & $c_{i+1}$ & $\hat r_i$ \\
        \hline
        0 & 0 & 0 & 0 & 0 \\
        \hline
        0 & 0 & 1 & 0 & 1 \\
        \hline
        0 & 1 & 0 & 0 & 1 \\
        \hline
        0 & 1 & 1 & 1 & 0 \\
        \hline
        1 & 0 & 0 & 0 & 1 \\
        \hline
        1 & 0 & 1 & 1 & 0 \\
        \hline
        1 & 1 & 0 & 1 & 0 \\
        \hline
        1 & 1 & 1 & 1 & 1 \\
        \hline
    \end{tabular}
\end{center}

We know that $c_{i+1} = (\hat x_i \wedge \hat y_i) \vee (\hat x_i \wedge c_i)
\vee (\hat y_i \wedge c_i)$ and we ALSO know that $\hat r_i = \hat x_i \oplus
\hat y_i \oplus c_i$. Thus ends the middle bit of the algorithm, so now let's
look at the whole algorithm.

Basically, we just string together a load of the components that we had before.
That is to say, if we have $n$ bits, we need $n$ of those adders. Each adder
takes 3 inputs: the $n$th digit of $x$ and $y$, and a carry, which comes from
the $n-1$th adder. At the very end of the block, we get a carry out.
Additionally, at each adder, we get the $n$th digit of the result ($\hat r$).
This could kinda be obvious, but if you think that, shut up. It's interesting to
know that there is not computation other than that in the adders. This is called
a \textbf{ripple carry adder} and it relates to the loop within the algorithm.
Each one of the `components' that I was talking about is called a \textbf{full
adder}, but we might replace this with a half adder later (we definitely will).
We can write this whole thing in forms of boolean expressions now, so there you
go.

Before we finish with these bad boys, we need to explore some examples. Let's
look at when $\hat x = 1111$ and $\hat y = 0001$. If we want to add those
together, we get:
\begin{align*}
    &\hat x = 1 \quad 1 \quad 1 \quad 1 \mapsto 15_{10} \\
    + \quad &\hat y = 0 \quad 0 \quad 0 \quad 1 \mapsto 1_{10} \\
         &\rule{75pt}{1pt}\\
         &c = 1 \quad 1 \quad 1 \quad 0 \quad (c_o = 1)\\
    &\hat r = 0 \quad 0 \quad 0 \quad 0
\end{align*}

This is an issue because $15 + 1$ is NOT 0, and this is an error. We use the
carry out to determine whether there has been an error, because if there is a
carry out of 1, then there is clearly an error.

If we look at the case when we use all 1s with 2's complement (i.e. -1), and we
add 1 to it, happily we get 0. We have the same behaviour, but the result is
right. Unfortunately, we lost the functionality of the 1 as a carry out error
marker, because there are still the possibility for errors. Take, for example,
$x = 0111_2 \mapsto 7_{10}$ and $y = 0001_2 \mapsto 1_{10}$. When we add these in
binary, we end up with $1000_2 \mapsto -8_{10}$, with a carry out of 0. This is
not the right answer. There is kind of a way around this, where if there are a
mismatch between the first bit of x and y and the first (most significant) bit
of c and r. Formally then, the sign of $\hat x, \hat y, \hat r$ should not end
up with a $+ve + +ve \rightarrow -ve$ and vice versa. This is known as a
\textit{carry error}. 

With the errors that we have detected, we should be responsible people and tell
the programmers that an error has occurred by medium of a flag, or even CORRECT
the error, (but we can't do that because we have a fixed number of bits).

\section{Transistors}%
\label{sec:Transistors}

An electrical current is a \textit{flow} of electrons. A capacitor (such as a
battery) works by having \textbf{free electrons} move from high to low
potential. A \textit{conductivity} rating says how easily electrons can move.
\begin{itemize}
    \item A conductor has \textit{high conductivity} and allows electrons to
        move easily.
    \item An insulator has low conductivity and does \textbf{not} allow
        electrons to move easily.
\end{itemize}

Silicon is a DOPE material, because there's lots of it, and it's also pretty
cheap. It's also inert (which means boring aka doesn't react in weird ways)
because it's stable enough to not react in weird ways with normal things
\textit{and} it can be doped with a donor material, which will allow us to
construct the materials with the precise sub atomic properties that we want.

The result of this is a semi conductor. `What is this?' I hear you ask. Well,
it's kind of a conductor and kind of not. If this isn't any clearer, here's a
little more info:
\begin{itemize}
    \item A \textbf{P-type} semi-conductor has extra holes, while
        \textbf{N-type} has extra electrons.
    \item if we sandwich together the P and N type layers together, the result
        is that the electrons can only move in one way. For example, from N to
        P, but not vice versa.
\end{itemize}

Back in the olden days, we used to have a vacuum tube, because when the filament
heats up, the electrons are produced into the vacuum, which are then attracted
by the plate. They're pretty reliable generally, but they fail a fail bit during
power on and off. It's also where we get the term \textit{bug} from, since a
literal bug could cause failures in this thing.

\subsection{MOSFETs}%
\label{sub:MOSFETs}
We're now going to look at MOSFETS gang. MOSFET stands for Metal Oxide
Semi-conductor Field Effect Transistor. Yeah, really. That's why there is an
abbreviation for it. A MOSFET has 3 parts, a \textbf{source}, \textbf{drain},
and a \textbf{gate}. The source and the drain are terminals, and the gate is
what controls the flow of electrons between the source and the drain. That, on a
simple level, is that, because any further description is pretty freaking
complicated and is not necessary for this course.

\subsubsection{N-MOSFET}%
\label{ssub:N-MOSFET}
An N-MOSFET (or negative MOSFET) is constructed from \textbf{n-type}
semiconductor terminals inside a p-type body. This means that applying a
potential difference to
the gate \textit{widens} the conductive channel, meaning that the source and
drain are connected, and the transistor is activated. Removing the potential
difference
\textit{narrows} the conductive channel and the source and the drain are
disconnected. Simply, \textbf{p.d. = current flows through, else block}.

\subsubsection{P-MOSFET}%
\label{ssub:P-MOSFET}
A P-MOSFET (or positive MOSFET) is constructed from \textbf{p-type}
semiconductor terminals inside an n-type body. Applying a potential difference to the gate
\textit{narrows} the conductive channel, meaning that no current can flow, and
removing the potential difference allows the current to flow. Simply, \textbf{p.d. = no
current flowing, else there is current flowing}. Also, p-types have a funny
looking bobble hat in a diagram.

These MOSFETS aren't normally used in isolation, and they are used in CMOS
cells, which stands for complimentary metal oxide semiconductor. We combine 2 of
one type and one of the other into one body, namely, the CMOS cell. It's pretty
useful because they work in complimentary ways, but there is also little
leakage, or \textbf{static} power consumption. It only consumes power during the
switching action (\textbf{dynamic} consumption).

\subsection{Manufacture}%
\label{sub:Manufacture}
It's necessary to be able to construct these bad boys in batch, because
otherwise we wouldn't be able to make big machines out of them because we need
so many of them. What we do is:
\begin{enumerate}
    \item Start with a clean, prepared \textbf{wafer}.
    \item Apply a layer of \textbf{substrate} material, such as a metal or a
        semi conductor.
    \item Apply a layer of photoresist. This material reacts differently when it
        is exposed to light.
    \item To do this, we expose a precise negative (or \textit{mask}) of design
        that hardens the exposed photoresist.
    \item Wash away the unhardened photoresist.
    \item Etch away the uncovered substrate.
    \item Strip away the hardened photoresist.
\end{enumerate}

Remember that this algorithm repeats over and over in order to make the result 3
dimensions, rather than 2. Regularity is a huge advantage because we can
manufacture a great number of similar components in a layer using a single
process. The feature size (it's 90nm big) relates to the resolution of the
process.

These components are USELESS in this form, so they're packaged before use, which
protects against damage, including heat sinks and an interface between the
component and the outside world using pins bonded to internal inputs and
outputs.

So, while MOSFETs are pretty great, there are down sides. Designing
complex functionality using transistors alone is really hard because
transistors are simply \textit{too} low level. We can address this problem
by repackaging groups of transistors into logic gates, since logic gates
are ordered logic gates such that we get certain functionality. Pretty
cool right? It's like nerdy Lego.

\section{Logic Gates}%
\label{sec:logic-gates}
If we form a \textbf{pull-up network} of P-MOSFET transistors, connected
to $V_{dd}$ (which is the high voltage rail), and a \textbf{pull-down
network} of N-MOSFETs, connected to $V_{ss}$ (the low voltage rail), and
assume that the power rails are everywhere:
\begin{align*}
    V_{ss} &= 0V \approx 0 \\
    V_{dd} &= 5V \approx 1
\end{align*}

We can then describe the operation of each logic gate using a truth table.

\subsection{NAND gate}%
\label{sub:nand}
% Insert picture of a nand gate I think
If both x and y are 0 (connected to $V_{ss}$), then:
\begin{enumerate}
    \item Both top P-MOSFETs will be connected.
    \item Both bottom N-MOSFETs will be disconnected.
    \item r (the output) will be connected to $V_{dd}$
\end{enumerate}

If x is 1 and y is 0, then:
\begin{enumerate}
    \item The right most P-MOSFET will be connected.
    \item The upper-most N-MOSFET will be disconnected.
    \item r will be connected to $V_{dd}$
\end{enumerate}

If x is 0 and y is 1, then:
\begin{enumerate}
    \item The left most P-MOSFET will be connected.
    \item The lower-most N-MOSFET will be disconnected.
    \item r will be connected to $V_{dd}$
\end{enumerate}

Finally, if both x and y are 1, then:
\begin{enumerate}
    \item Both top P-MOSFETs will be disconnected.
    \item Both bottom N-MOSFETs will be connected.
    \item r will be connected to $V_{ss}$
\end{enumerate}

\subsection{NOR gate}%
\label{sub:nor}
% Insert picture of a nor gate
If both x and y are 0, then:
\begin{enumerate}
    \item Both top P-MOSFETs will be connected.
    \item Both bottom N-MOSFETs will be disconnected.
    \item r will be connected to $V_{dd}$
\end{enumerate}

If x is 1 and y is 0, then:
\begin{enumerate}
    \item The upper-most P-MOSFET will be disconnected
    \item the left-most N-MOSFET will be connected.
    \item r will be connected to $V_{ss}$
\end{enumerate}

If x is 0 and y is 1, then:
\begin{enumerate}
    \item The lower-most P-MOSFET will be disconnected.
    \item The right-most N-MOSFET will be connected.
    \item r will be connected to $V_{ss}$.
\end{enumerate}

If both x and y are 1, then:
\begin{enumerate}
    \item Both top P-MOSFETs will be disconnected.
    \item Both bottom N-MOSFETs will be connected.
    \item r will be connected to $V_{ss}$.
\end{enumerate}

\subsection{Physical limitations}%
\label{sub:Physical limitations}
There are, of course, some physical limitations that we haven't discussed
yet. There are two classes of delay (often described as
\textbf{propagation delay}), that will dictate the time between change to
some input and corresponding change in an output. These are:
\begin{itemize}
    \item \textbf{Wire delay:} This relates to the time taken for the
        current to move through the conductive wire from one point to
        another.
    \item \textbf{Gate delay:} This relates to the time taken for the
        transistors in each gate to switch between the connected and
        disconnected states
\end{itemize}
Normally, the gate delay is more than wire delay, and both relate to the
implementations. Gate delay is the fault of the properties of the
transistors used, and wire delay to the properties of the wire.

\textbf{Critical path} is the longest sequential delay possible in some
combinatorial logic. Basically, the worst case scenario in a given logic
circuit.

Ideally, we'd get a perfect digital on/off graph of the response, while in
reality, we get a more curved graph. It would also make sense to have 1
and 0 as a threshold, rather than a definitive voltage. This fuzzy
representation allows for some inaccuracies that are unavoidable with this
physical implementation.

Including gate delay gives us a \textit{dynamic} view computation. We're
gonna kind of ignore wire delay for the moment and arbitrarily choose some
delays for the gates:
\begin{enumerate}
    \item NOT: 10ns
    \item AND: 20ns
    \item OR: 20ns
\end{enumerate}

If we then did some computation, we could see the output changing after we
switch x from 0 to 1 and keep y as 1. If we're using an XOR gate, then it
would take 50ns to completely change to 0 (since x and y are both 1, hence
x XOR y = 0).

It is cool to use \textit{3-stage logic}, using an extra value:
\begin{itemize}
    \item 0 is false
    \item 1 is true
    \item Z is \textbf{high impedance}
\end{itemize}

High impedance is the null value, so we can allow a wire to be
disconnected.

If we have two inputs connected to an output, you can have some weird
results where the two inputs are in conflict with the other. The way to
fix this is to have some enable switch on the two inputs to the output, so
we don't get any inconsistencies.

Here are some more definitions:
\begin{itemize}
    \item \textbf{fan-in} is used to describe the number of inputs to a
        given gate.
    \item \textbf{fan-out} is used to describe the number of inputs (or
        \textit{other} gates) the output of a given gate is connected to.
\end{itemize}

\section{Combinatorial Logic}%
\label{sec:combinatorial}

The logic gates that we've discussed already are higher level than the
transistors that we started off with. Armed with these new logic gates, we
can move onto \textit{actual} high level components. We want components
that are closer to what we can actually do things with.

Thus far, we've just looked at various ways of writing the same things;
having moved from the completely abstract pencil and paper, to the
implementation of logic gates and NAND-boards.

\subsection{Design patterns}%
\label{sub:patterns}
There are a number of design patterns that we can take:

\subsubsection{Decomposition}%
\label{ssub:Decomposition}

Divide and conquer, take some complicated design, and break it down into
simpler and more manageable parts.

\subsubsection{Sharing}%
\label{ssub:Sharing}

We can replace two AND gates that exist in some design with a single AND
gate, using the same gate from the usage points.  It makes sense because
the output will always be the same.

\subsubsection{Isolated Replication}%
\label{ssub:isolated}

Say we have a 2 input AND gate, and we want a 2-input \textit{m}-bit AND
gate, which is simply a replication of 2-input, 1-bit AND gates, then we
are going to have \textit{m}-bits of the AND gate. We're also saying that
they are operating in parallel, and that each bit does not affect the
neighbouring bits.

\subsubsection{Cascaded Replication}%
\label{ssub:cascaded}

Using the same example as before, instead of using \textit{m} AND gates,
we want something like this:
\begin{alignat*}{4}
    r &= & &x_0 \wedge x_1   &&\ \wedge \ & &x_2 \wedge x_3 \\
      &= (& &x_0 \wedge x_1) &&\wedge \ (& &x_2 \wedge x_3)
\end{alignat*}

You can also write that like this:
\begin{equation*}
    r = \bigwedge^{n-1}_{i=0} x_i
\end{equation*}

It's different from isolated replication, because the and gates are
entirely isolated from one another.

\subsection{Mechanical Derivation}%
\label{sub:mechanical}
In every case, the process is the same. We want to be able to process some
truth table and get something that will work (with a boolean expression).
So, let's let $T_i$ denote the $j$th input for $0 \le j < n$, and let $O$
denote the single output.

\begin{enumerate}
    \item Find a set $T$ such that $i \in T$ iff. $O = 1$ in the $i$th row
        of the truth table.
    \item For each $i \in T$, form a term $t_i$ by ANDing together all the
        variables while following two rules:
        \begin{enumerate}
            \item if $T_j = 1$ in the $i$th row, then we use $T_j$ as is,
                but
            \item If $T_j = 0$ in the $i$th row, then we use $-T_j$
        \end{enumerate}
    \item An expression implementing this function is then formed by ORing
        all of the terms together:
        \begin{equation*}
            e = \bigvee_{i \in T} t_i
        \end{equation*}
\end{enumerate}

Let's consider the example of deriving an expression for XOR:
\begin{equation*}
    r = f(x,y) = x \oplus y
\end{equation*}

This is a function described by the following table:
\begin{center}
    \begin{tabular}{|c c|c|}
        \hline
        $x$ & $y$ & $r$ \\
        \hline
        0 & 1 & 1 \\
        \hline
        1 & 0 & 1 \\
        \hline
        1 & 1 & 0 \\
        \hline
    \end{tabular}
\end{center}

\subsection{Karnaugh Map}%
\label{sub:karnaugh}

The simple algorithm for creating a Karnaugh map is as follows:
\begin{enumerate}
    \item Draw a rectangular ($p \times 1$) element grid:
    \begin{enumerate}
        \item $p = q = 0 (mod 2)$, and
        \item $p \cdot q = 2^n$
    \end{enumerate}
    \item Fill the grid elements with the output that corresponds to
        inputs for that row and column.
    \item Cover rectangular groups of
        adjacent 1 elements which are of total size $2^n$ for some group
        $m$, groups can `wrap around' if you like, so they can go over
        edges of the grid and overlap.
    \item Translate each group into a single term in some SoP form Boolean
        expression, where
    \begin{enumerate}
        \item bigger groups
        \item less groups
    \end{enumerate}
    mean a simpler expression.
\end{enumerate}

The basic idea is that we don't count up as per normal in binary, but we
actually count up by changing one binary digit at a time. You then group
them, and the groups have to be adjacent (but they can't form a kind of L
shape). After this, it's easy to group them into some format.

\subsection{Building blocks}%
\label{sub:blocks}

There are two more blocks that we're going to look at now:
\begin{enumerate}
    \item \textbf{Multiplexer}
        \begin{itemize}
            \item It has $m$ inputs,
            \item 1 output and
            \item uses a ($\log_2(m)$)-bit control signal input to choose
                which input is connected to the output.
        \end{itemize}
    \item \textbf{Demultiplexer}
        \begin{itemize}
            \item It has 1 input,
            \item $m$ outputs and
            \item uses a ($\log_2(m)$)-bit control signal to choose which
                output is connected to the input
        \end{itemize}
\end{enumerate}

Remember that the inputs and outputs are $n$ bits, but they obviously have
to match up. The connection that is made is continuous, because both
components are \textit{combinatorial}.

\subsubsection{Multiplexer}%
\label{ssub:Multiplexer}
The behaviour of a 2-input, 1-bit multiplexer is as follows:
\begin{equation*}
    r = (\neg \ c \wedge x) \vee (c \wedge y)
\end{equation*}

It can be more clearly represented by this table:
\begin{center}
    \begin{tabular}{|c c c|c|}
        \hline
        c & x & y & r \\
        \hline
        0 & 0 & ? & 0 \\ \hline
        0 & 1 & ? & 1 \\ \hline
        1 & ? & 0 & 0 \\ \hline
        1 & ? & 1 & 1 \\ \hline
    \end{tabular}
\end{center}


\subsubsection{Demultiplexer}%
\label{ssub:Demultiplexer}

The behaviour of a 2-output, 1-bit demultiplexer is as follows:
\begin{align*}
    r_0 &= \neg \ c \wedge x \\
    r_1 &- c \wedge x
\end{align*}

It can be more clearly represented in this table:
\begin{center}
    \begin{tabular}{|c c|c c|}
        \hline
        $c$ & $x$ & $r_1$ & $r_0$ \\ \hline
        0 & 0 & ? & 0 \\ \hline
        0 & 1 & ? & 1 \\ \hline
        1 & 0 & 0 & ? \\ \hline
        1 & 1 & 1 & ? \\ \hline
    \end{tabular}
\end{center}

Multiplexers can be used to for isolated replication. If we line up 4
multiplexers, we get a control signal in to choose between two choices,
and we get one of those out, meaning that we end up with a 4-bit output.
We can scale this in the same way as before and use the \textit{cascaded}
pattern, meaning that they interact with one another. We now need a 2-bit
control signal however, because in the other design, each multiplexer uses
the same control signal.

Building on this idea, we can look at two more components:
\begin{enumerate}
    \item Half adder
        \begin{itemize}
            \item Has 2 inputs, $x$ and $y$,
            \item Computes the 2-bit result $x + y$
            \item Has 2 outputs: a sum $s$ and a carry out $co$ (which are
                the least significant bit and the most significant bit of
                the result)
        \end{itemize}
    \item Full adder
        \begin{itemize}
            \item Has 3 inputs: $x$, $y$, and a carry-in $ci$
            \item Computes the 2-bit result $x + y + ci$
            \item Has 2 outputs: a sum $s$ and a carry out $co$ (which are
                the least significant bit and the most significant bit of
                the result)
        \end{itemize}
\end{enumerate}

\subsection{Half adder}%
\label{sub:half}
The behaviour of the half adder looks like this:
\begin{center}
    \begin{tabular}{|c c|c|c|}
        \hline
        $x$ & $y$ & $co$ & $s$ \\ \hline
    \end{tabular}
\end{center}

And can be displayed by the following equations:
\begin{align*}
    co &= x \wedge y \\
    s &= x \oplus y
\end{align*}

\subsection{Full adder}%
\label{sub:full}
The behaviour of a full adder looks like this:
\begin{center}
    \begin{tabular}{|c c c|c|c|}
        \hline
        $ci$ & $x$ & y & $co$ & $s$ \\ \hline
        0 & 0 & 0 & 0 & 0 \\ \hline
        0 & 0 & 1 & 0 & 1 \\ \hline
        0 & 1 & 0 & 0 & 1 \\ \hline
        0 & 1 & 1 & 1 & 0 \\ \hline
        1 & 0 & 0 & 0 & 1 \\ \hline
        1 & 0 & 1 & 1 & 0 \\ \hline
        1 & 1 & 0 & 1 & 0 \\ \hline
        1 & 1 & 1 & 1 & 1 \\ \hline
    \end{tabular}
\end{center}

And can be displayed by the following equations:
\begin{align*}
    co &= (x \wedge y) \vee (x \wedge ci) \vee (y \wedge ci) \\
       &= (x \wedge y) \vee ((x \oplus y) \wedge ci) \\
    s &= x \oplus y \oplus ci
\end{align*}

A full adder is kind of like two half adders put together in a nice way.

There is a circuit for the full adder that allows us to implement the
cascading design choice to replicate the $n$-bit addition; cascading
because of the carries that we have are \textit{cascaded} into one
another. It might not be immediately clear that it's cascaded when we
initially looked at it, but now as we look at it from a higher value, then
we can clearly see that it is indeed cascaded.


Moving onto equality comparators, we have 2 final building blocks:
\begin{enumerate}
    \item An equality comparator
        \begin{itemize}
            \item Has 2 inputs: $x$ and $y$,
            \item computes the 1 output as:
                \begin{equation*}
                    r = \bigg \{
                   \begin{matrix}
                       1 & \text{if } $x = y$ \\
                       0 & \text{otherwise}
                   \end{matrix}
                \end{equation*}
        \end{itemize}
    \item A less-than comparator
        \begin{itemize}
            \item Has 2 inputs: $x$ and $y$,
            \item computes the 1 output as:
                \begin{equation*}
                    r = \bigg \{
                        \begin{matrix}
                            1 & \text{if } x < y \\
                            0 & \text{otherwise}
                        \end{matrix}
                \end{equation*}
        \end{itemize}
\end{enumerate}

All of the inputs and outputs are only 1 bit.

Even though we only have these two, we can do a lot of other comparators,
for example, we can get $x \neq y$ from passing the result of an equality
gate through a not gate.

\subsection{Equality comparator}%
\label{sub:equality}
The behaviour of the equality comparator is as follows:
\begin{equation*}
    r = \neg (x \oplus y)
\end{equation*}

And can be modelled by this table:
\begin{center}
    \begin{tabular}{|c c|c|}
        \hline
        $x$ & $y$ & $r$ \\ \hline
        0 & 0 & 1 \\ \hline
        0 & 1 & 0 \\ \hline
        1 & 0 & 0 \\ \hline
        1 & 1 & 1 \\ \hline
    \end{tabular}
\end{center}

\subsection{Less-than comparator}%
\label{sub:less-than}

The behaviour of the less-than comparator can be modelled as follows:
\begin{equation*}
    r = \neg x \wedge y
\end{equation*}

And can be shown in this table:
\begin{center}
    \begin{tabular}{|c c|c|}
        \hline
        $x$ & $y$ & $r$ \\ \hline
        0 & 0 & 0 \\ \hline
        0 & 1 & 1 \\ \hline
        1 & 0 & 0 \\ \hline
        1 & 1 & 0 \\ \hline
    \end{tabular}
\end{center}

In the same way that we had a ripple carry adder, we can have an $n$-bit
comparison. For a 3-bit number, for them to be equal, every single bit
must match. Therefore, if we first compare the first bits, and then and
this with the second bits, and then and it again this with the third bits,
this will give us the result we require.

For a cyclic less-than format, we have to do use a less than and an
equality block to be able to go through and work out which is bigger.
Let's say we take two numbers, $x$ and $y$, where $x = 123$ and $y = 223$.
We can see that $y$ is bigger than $x$, because we started at the left
most digit and compared these. This is quite an easy example, so let's
look at another one.

In this example, $x = 121, \ y = 123$. Here, we start at the left most
digit, which are the same. So, we move inward, which is the same again.
Doing this one more time, we can see that this digit of $y$ is bigger than
that of $x$, therefore $y > x$. We have some rules here then:
\begin{enumerate}
    \item If $x_i < y_i$, then $x < y$
    \item If $x_i > y_i$, then $x \not < y$
    \item if $x_i = y_i$, then $x < y$ IF the $rest(x) < rest(y)$
\end{enumerate}

In the format of an equation:
\begin{align*}
    x<y \text{if } (x_i < y_i) \vee (x_i = y_i \wedge \text{ rest } x <
    \text{ rest } y)
\end{align*}

We have this joined up to each bit of the input.

The final set of components are the \textbf{encoder} and the
\textbf{decoder}. They can also be viewed as translators, or, formally:
\begin{enumerate}
    \item an $n$-to$m$ encoder translates an $n$-bit input into some
        $m$-bit code word, and
    \item an $m$-to-$n$ decoder translates an $m$-bit code word
        \textbf{back} into the same $n$-bit output.
\end{enumerate}

Where if only one output is allowed to be 1 at a time, we call it a
\textbf{one of many} encoder.

A \textit{general} building block is impossible because it depends on the
scheme for encoding or decoding. Let's look at an example:
\begin{enumerate}
    \item To encode, take $n$ inputs, for example $x_i$ for $0 \le i < n$
        and produce an unsigned integer $x'$ that determines which $x_i =
        1$
    \item To decode, take $x'$ and set the right $x_i = 1$
\end{enumerate}

Where for all $j \not = i, x_j' = 0$ (so both are \textbf{one-of-many})
and this means we recover the original $x$.

We can map the behaviour by creating a truth table:
\begin{center}
    \begin{tabular}{|c c c c|c c|}
        \hline
        $x_3$& $x_2$ & $x_1$ & $x_0$ & $x_1'$ & $x_0'$ \\
        \hline
        0 & 0 & 0 & 1 & 0 & 0 \\
        \hline
        0 & 0 & 1 & 0 & 0 & 1 \\
        \hline
        0 & 1 & 0 & 0 & 1 & 0 \\
        \hline
        1 & 0 & 0 & 0 & 1 & 1 \\
        \hline
    \end{tabular}
\end{center}
Which also correlates to:
\begin{align*}
    x_0' = x_1 \vee x_3 \\
    x_1' = x_2 \vee x_3
\end{align*}

The decoder should be exactly the opposite of this, and can be modelled by
this equation:
\begin{align*}
    x_0 = \neg x'_0 \wedge \neg x_1' \\
    x_1 = x_0' \wedge \neg x_1' \\
    x_2 = \neg x_0' \wedge x_1' \\
    x_3 = x_0' \wedge x_1'
\end{align*}

There is a problem with this because if we break the rules and set more
than one bit of $x$ to 1, then the encoder fails because it produces two
1s. The solution is to produce a priority encoder, where one input is
given priority over another. 
\begin{center}
    \begin{tabular}{|c c c c|c c|}
        \hline
        $x_3$& $x_2$ & $x_1$ & $x_0$ & $x_1'$ & $x_0'$ \\
        \hline
        0 & 0 & 0 & 1 & 0 & 0 \\
        \hline
        0 & 0 & 1 & ? & 0 & 1 \\
        \hline
        0 & 1 & ? & ? & 1 & 0 \\
        \hline
        1 & ? & ? & ? & 1 & 1 \\
        \hline
    \end{tabular}
\end{center}

\section{Sequential logic}%
\label{sec:sequential}

Imagine that we need a cyclic $n$-bit counter, such as a component whose
output steps through the values:
\begin{equation*}
    0,1,\cdots,2^n -1, 0,1,\cdots
\end{equation*}

But is otherwise free running.

We already know how to make an $n$ bit adder, so we can just compute $r
\leftarrow r + 1$ over and over. It sounds cool, but we can't initialise the
value, and also we don't let the output of each full-adder settle before
it's used again as an output. 

The main issue that we have is that combinatorial logic has some
limitations because we can't control \textit{when} a design computes some
output, nor remember the output when produced. A good solution is
\textbf{sequential logic} which demands that there is some way to control
components, one or more components that remember the state that they're
in, and a mechanism to perform computation as a sequence of steps, rather
than continuously.

\subsection{Clocks}%
\label{sub:Clocks}

A clock is a signal that alternates between 1 and 0. We call the period of
time where it is 1 as \textit{positive level}, and when it has a value of
0, it has a \textit{negative level}. Also, when it changes from positive
to negative, we call this a \textbf{negative edge}, while the opposite of
this is called a \textbf{positive edge}.

We want to clock to \textit{trigger} events in order to synchronise
components within a design, while the clock frequency is how many clock
cycles happen in a unit of time. It's gotta be fast enough that the design
goals are met, but slow enough that the critical path of a given step is
not exceeded. The faster the clock ticks, the better things are, but we
can't go too fast. The clock is therefore a conductor.

An $n$ phase-clock is distributed as $n$ separate signals along $n$
separate wires. A 2-phase instance is really useful because it has the
features of a 1-phase clock, such as the clock period etc, but there is a
guarantee that positive levels of the two clocks don't overlap, and the
behaviour is parameterisable by altering the clock length ($\delta_i$).

\subsection{Latches and flipflops}%
\label{sub:Latches}

A bistable component can exist in two stable states, like 0 or 1 at a
given point. It can:
\begin{itemize}
    \item Retain some \textbf{current state} -- Q --, which can also be read as
        output, and
    \item be updated to some \textbf{next state} -- Q' -- which is provided as an
        input.
\end{itemize}

Under control of an enable signal \textit{en}. We say it is:
\begin{itemize}
    \item \textbf{Level-triggered} and hence a latch if it's updated by a
        given level of \textit{en}, or
    \item \textbf{Edge triggered} and hence a flipflop if it's updated by
        a given edge of \textit{en}.
\end{itemize}

They both have different properties, but both are useful.

\subsubsection{SR component}%
\label{ssub:SR}
An "SR" latch/flip-flop component has two inputs: S (for set), and R (for
reset). When enabled, the following behaviour can be observed:
\begin{itemize}
    \item S = 0, R = 0, the component retains Q
    \item S = 1, R = 0, the components updates to Q = 1,
    \item S = 0, R = 1, the component updates to Q = 0,
    \item S = 1, R = 1, the component is metastable.
\end{itemize}

When it's not enabled, the component is in `storage mode'

It can also be described by the truth table:
\begin{center}
    \begin{tabular}{|c c|c c|c c|}
        \hline
        $S$ & $R$ & $Q$ & $\neg Q$ & $Q'$ & $\neg$ $Q'$ \\ \hline
        0 & 0 & 0 & 1 & 0 & 1 \\ \hline
        0 & 0 & 1 & 0 & 1 & 0 \\ \hline
        0 & 1 & ? & ? & 0 & 1 \\ \hline
        1 & 0 & ? & ? & 1 & 0 \\ \hline
        1 & 1 & ? & ? & ? & ? \\ \hline
    \end{tabular}
\end{center}

\subsubsection{D component}%
\label{ssub:D}
A "D" latch/flip-flop component has a single input - $D$. When it's
enabled, the following behaviour can be observed:
\begin{itemize}
    \item $D$ = 1, the component updates to $Q$ = 1
    \item $D$ = 0, the component updates to $Q$ = 0
\end{itemize}

When it's not enabled, the component is in storage mode, and retains $Q$.
The behaviour can be modelled by this truth table:
\begin{center}
    \begin{tabular}{|c|c c|c c|}
        \hline
        $D$ & $Q$ & $\neg Q$ & $Q'$ & $\neg Q'$ \\ \hline
        0 & ? & ? & 0 & 1 \\ \hline
        1 & ? & ? & 1 & 0 \\ \hline
    \end{tabular}
\end{center}

How do we design a simple SR latch? We use two \textit{cross-coupled} NOR
gates. It can be shown through these equations:
\begin{align*}
    S \bar \wedge Q = \neg Q \\
    R \bar \wedge \neg Q = Q
\end{align*}

There's a loop here, because in order to know the output from the top NOR
gate, we need to know the output from the bottom one, and vice versa. So
what gives?? Well, it's alright if we look at the behaviour of a NOR gate.

We know that a NOR gate gives us a 1 only when both inputs are 0. If
either input is 1, then NOR will produce a 0. This is really important for
our gate to actually function.

If we set $S$ = 0, then the output from the top NOR gate is forced to
produce 0. Then, if $R$ is 0, then the output is 1. Similarly, if $R$ is 0,
the output for THAT NOR gate is 1. Therefore, we get what we want. The
reverse is true when $R$ is 1 and $S$ is 0.

There is some form of issue, though, because if both $R$ and $S$ are 0,
then either the top NOR gate spits out 1, or the bottom NOR gate does.
That's just the way it is. There's no logic error, it's just a bit weird.

There's just one case left to study. If $S$ and $R$ are 1. In the truth
table, we stated that we didn't care what happened, but in the
implementation, then both $Q$ and $\neg Q$ are both 0, which is clearly a
bloody issue because $Q$ cannot equal $\neg Q$. What happens now? Well,
the latch has to settle in some case, either $Q = 1$ or $Q = 0$. We have 0
control over what happens. We don't want this behaviour at all. This is
what meta stability means. It's stable in some sense, but kind of not.

\subsubsection{Updates}%
\label{ssub:Updates}
We want to be able to control when updates occur, because without this,
it's not really a latch yet. To do this, we have an AND gate before we get
to the criss-cross NOR gated latch, so we AND S and the enable signal, and
R and the enable signal, and then pass the results from this into the old
S and R respectively. This is a technique known as \textbf{gating}.

Ok, this just in. WE CAN GET RID OF THE ISSUE OF META-STABILITY. Yep, you
heard that right. We force $R = \neg S$, so we get either $S = 0, R = 1$
or $S = 1, R = 0$. Dope, right?

The difference between a latch and a flip-flop is that is has a little
triangle on the symbol, indicating that it's edge triggered.

For a latch, the point value of $Q$ only changes when the enable signal is
1, it then matches the input $D$. Some people call this a `transparent'
latch, because it's like the latch isn't even there.

In a flip flop, the input only changes to the input $D$ on the edges of
the clock (when the enable switch is on).

\subsubsection{Registers}%
\label{ssub:Registers}
We normally group these components into \textbf{registers}, which is an
$n$ bit register that can then store an $n$ bit value. We take each latch
as a single bit.

\subsection{Returning to the original issue}%
\label{sub:original}
So we have some data path of computation and/or storage components, and a
control path, that tells the components in the data path what to do and
when to do it.

What we have is some latch based register, that takes in $\phi_1$, the
first clock signal as an enable signal. This gives an input to some
combinatorial logic, which then passes the result into another latch based
register that takes in $\phi_2$, the second clock signal. This register
passes the information back to the first register. And so on. This creates
a kind of buffer that gives us a counter that we wanted. The two registers
mean that the loop is broken, because it is impossible for both registers
to be enabled at the same time; breaking the infinite loop that we would
have without them.

The combinatorial logic in the middle is just a big boy ripple carry
adder. There is also a reset signal that is NOTED, and then ANDED with the
individual bits of the ripple carry adder, meaning that when the RESET
signal is 1, it becomes 0, thereby resetting all bits, because $x \wedge 0
= 0$.

\subsection{Memory Component}%
\label{sub:memory}

An $n$-bit register based on latches or flip-flops has a couple of
limitations:
\begin{enumerate}
    \item Each latch or flip-flop in the register needs a fairly big
        number of transistors. This limits the viable capacity
    \item The register is also not addressable, because an address or
        index allows dynamic rather than static reference to some stored
        data. That means that we have something like variables, when we
        want an array.
\end{enumerate}

The solution to these limitations is to have a memory component. The
memory is connected to a \textbf{user} which could be a CPU by control
signals, a data bus, and an address bus. Sound familiar? We'll also say
that the memory has a capacity of $n = 2^{n'}$ addressable words and each
word is $w$ bits, where $n \gg w$.

There are a number of ways to classify a given memory component:
\begin{enumerate}
    \item Volatility
    \begin{itemize}
        \item \textbf{Volatile} means that the content is lost when the
            component is powered off
        \item \textbf{Non volatile} means that the content is not lost
            when the component is powered off
    \end{itemize}
    \item Interface type:
    \begin{itemize}
        \item \textbf{Synchronous} where a clock or pre-determined timing
            information synchronises steps
        \item \textbf{Asynchronous} where a protocol synchronises steps
    \end{itemize}
    \item Access type
    \begin{itemize}
        \item \textbf{Random vs. Constrained} access to content.
        \item \textbf{Random Access Memory (RAM)} which can be read from
            \textit{and} written to.
        \item \textbf{Read Only Memory (ROM)} which can only be read to
    \end{itemize}
\end{enumerate}

We're going to focus on a \textbf{volatile, synchronous RAM}.

\subsubsection{History of memory}%
\label{ssub:history-mem}

In the olden days, we had \textbf{delay line} memory, that contained
mercury. There is a speaker at one end to store sound waves into the line
and a microphone at the other end to read them out.

Values are stored in the sense that the corresponding waves take some time
to propagate, when they get to one end, they are either replaced, or fed
back into the other.

This is known as sequential access because you need to wait for the data
you want to appear.

After delay line came \textbf{magnetic-core} memory, where the basic idea
is that the memory is a matrix of small magnetic rings or cores which can
be magnetically polarised to store values. Wires are threaded through the
cores to control them (so to read or write values). The magnetic
polarisation is retained, so this is non-volatile. Sometimes, main memory
is termed \textbf{core-memory} (or \textbf{core dump}), which is in
reference to this.

\subsubsection{Low-level implementation}%
\label{ssub:Low-level implementation}

There are two kinds of RAM that we can use: \textbf{Static RAM} or (SRAM)
and \textbf{Dynamic RAM} or (DRAM).

SRAM is:
\begin{itemize}
    \item manufacturable in lower densities 
    \item More expensive to manufacture
    \item Faster to access
    \item Easier to interface with
    \item Ideal for latency optimised contexts (like cache memory)
\end{itemize}

DRAM is:
\begin{itemize}
    \item Manufacturable in higher densities.
    \item Less expensive to manufacture
    \item Slower to access
    \item Harder to interface with
    \item Ideal for capacity optimised contexts (like main memory)
\end{itemize}

Simply, an \textbf{SRAM cell} resembles two NOT gates on the inner loop.
It basically carries around a signal that is reinforced by the NOT gates.
On the outside, there are two transistors that are connected to some
control signal $wl$. The transistors allow access to the values inside the
loop. To read from this, we pre-charge the transistors to 1. This is
logically inconsistent, but it means that we allow one of them to be
overridden after we allow access to the inside of the loop (through $wl$).
To write to this, we charge the left transistor (or $bl$) to the value,
and the right one to the opposite (since it is called $\neg bl$).

An issue that we might find is different signal strengths. We're going to
kind of brush it under the carpet and not think about it.

In total, we'd need 6 transistors: 2 for each of the 2 NOT gates, and one
each for the two on the outside. Therefore, it's known as a \textbf{6t}
SRAM cell. It's loads less than the 20 or so required by the latches and
flip-flops. This therefore solves one of the limitations of the latches.

A \textbf{DRAM} cell is constructed only using 1 single transistor and a
capacitor (kind of like a battery because it stores charge). The
transistor again has an enable signal $wl$, and some input $bl$. If we
want to read, then we set $bl$ to 1. If the contents is 1, then there is a
discharge through $bl$, else there will not be.  To be able to write, we
set $bl$ to the required value, and then set $wl$ to 1. After this, the
capacitor is charged with the required value.

The issue with this cell is that the capacitor can only do limited
charges. Also, the same issue as before, because the signal strength might
not be enough for the required circuit. Additionally, the charge stored in
the capacitor will decay over time. We need some refresh thing in the
background to ensure that the value is not lost over time.  Reading from a
DRAM cell means that the value is destroyed, so we need to rectify that
through the refresh mechanism. The latency is long from a DRAM cell
because the capacitor has a long access time.

\subsubsection{Higher level implementation}%
\label{ssub:high-level}
A \textbf{memory device} is constructed from (roughly) three components:
\begin{enumerate}
    \item A \textbf{memory array} (or matrix) of replicated cells with
        \textit{r} rows and \textit{c} columns.
    \item A \textbf{row decoder} which given an address (de)activates
        associated cells in that row, and
    \item A \textbf{column decoder} which given an address (de)selects
        associated cells in that column
\end{enumerate}

Along with additional logic to allow use:
\begin{enumerate}
    \item \textbf{Bit line conditioning} to ensure that the bit lines are
        strong enough to be effective,
    \item \textbf{Sense amplifiers} to ensure that the output from the
        array is usable.
\end{enumerate}

To access these bad boys, we get an input to the row decode from the
address bus, and an output from the column decode. It's kind of like a
multiplexer and then a demultiplexer.

The difference between DRAM and SRAM is simply that the cells in the
middle are DRAM rather than SRAM.

DRAM also has buffers (both for the row and the column). 

\section{Finite State Machines}%
\label{sec:FSMs}

\subsection{Automata Theory}%
\label{sub:automata}

First things first, we need some definitions:
\begin{itemize}
    \item \textbf{Alphabet}: a non-empty set of symbols
    \item \textbf{String}: with respect to some alphabet $\Sigma$ is a
        sequence of finite length, whose elements are members of $\Sigma$.
    \item \textbf{Language}: a set of strings.
\end{itemize}

Finite state machines are a model of computation. What does this mean?
It's a set of allowable (or possible) operations. It is basically an
idealised computer that is in some finite set of states at a given point
in time. It accepts an input string (with respect to some alphabet) one
symbol at a time. Each symbol produces a change in state. Once it's run
out of inputs, the computer stops. Depending on the state it stops in, it
can either \textit{accept} or \textit{reject} the string. For a language
of all possible strings that the computer could accept, the
computer \textbf{accepts} the language.

Finite state machines do not have any memory. The more advanced machine
that you have, the more memory you have. We're looking at a kind of
mid-rate deal, that does not have any memory.

For a more formal definition of FSMs, it is a tuple:
\begin{equation*}
    C = (S,s,A,\Sigma,\Gamma,\delta,\omega)
\end{equation*}

With the following definitions:
\begin{itemize}
    \item $S$ is a finite set of states that includes a \textbf{start
        state}, $s \in S$
    \item $A \subseteq S$, a finite set of \textbf{accepting states}
    \item An \textbf{input alphabet} $\Sigma$ and an \textbf{output
        alphabet} $\Gamma$
    \item A \textbf{transition function}:
        \begin{equation*}
            \delta : S \times \Sigma \rightarrow S
        \end{equation*}
    \item An \textbf{output function}:
        \begin{equation*}
            \omega : S \rightarrow \Gamma
        \end{equation*}
        In the case of a \textbf{Moore FSM}, or
        \begin{equation*}
            \delta : S \times \Sigma \rightarrow \Gamma
        \end{equation*}
        In the case of a \textbf{Mealy FSM} 
\end{itemize}

An empty input is called $\epsilon$

We need to design an FSM that decides whether a binary sequence $X$ has an
odd number of 1 elements in it. The \textit{accepting state} is
$S_{\text{odd}}$.

For example, say we have the input string $X = \langle 1,0,1,1 \rangle$
the transition are:
\begin{equation*}
    \rightarrow S_{even} \xrightarrow{X_0 = 1} S_{odd} \xrightarrow{X_1 =
0} S_{odd} \xrightarrow{X_2 = 1} S_{even} \xrightarrow{X_3 = 1}
    S_{odd}
\end{equation*}

So the input is accepted because there are an odd number of 1 elements.

The real world example of this is to use a regular expression (or Regex).

\subsection{FSMs in hardware}%
\label{sub:hardware-fsm}

Using an FSM with a latch based implementation is that:
\begin{enumerate}
    \item The state is retained in a register
    \item $\delta$ and $\omega$ are simple combinatorial logic
    \item Within the current clock cycle:
        \begin{enumerate}
            \item $\omega$ computes the output from the current state and
                input
            \item $\delta$ computes the next state from the current state
                and input
        \end{enumerate}
    \item The next state is latched by an appropriate feature in the
        clock.
\end{enumerate}

This is a computer that we can actually build now!

We have two options as to the choice of encoding the state of the machine:

There is \textbf{binary encoding}:
\begin{align*}
    S_0 \mapsto \langle 0, 0, 0 \rangle \\
    S_1 \mapsto \langle 1, 0, 0 \rangle \\
    S_2 \mapsto \langle 0, 1, 0 \rangle \\
    S_3 \mapsto \langle 1, 1, 0 \rangle \\
    S_4 \mapsto \langle 0, 0, 1 \rangle \\
    S_5 \mapsto \langle 1, 0, 1 \rangle 
\end{align*}

Or, there is \textbf{One-hot encoding}:
\begin{align*}
    S_0 \mapsto \langle 1, 0, 0, 0, 0, 0 \rangle \\
    S_1 \mapsto \langle 0, 1, 0, 0, 0, 0 \rangle \\
    S_2 \mapsto \langle 0, 0, 1, 0, 0, 0 \rangle \\
    S_3 \mapsto \langle 0, 0, 0, 1, 0, 0 \rangle \\
    S_4 \mapsto \langle 0, 0, 0, 0, 1, 0 \rangle \\
    S_5 \mapsto \langle 0, 0, 0, 0, 0, 1 \rangle 
\end{align*}

There is a trade off here because there is more space required for the
one-hot, but there is less energy required because we are only flipping
two bits at a time.

Now, we want to design an FSM that acts like a cyclic counter modulo $n$
(rather than $2^n$ as before). If $n = 6$ for example, we want a component
whose output $r$ steps through values:
\begin{equation*}
    0, 1, 2, 3, 4, 5, 0, 1, \cdots
\end{equation*}

The first couple of steps of our algorithm state that we need to
enumerate each state, and then give them some abstract label. We can
represent this using a table:

\begin{center}
    \begin{tabular}{|c|c|c|}
        \hline
        & $\delta$ & $\omega$ \\ \hline
        $Q$ & $Q'$ & $r$ \\ \hline
        $S_0$ & $S_1$ & 0 \\ \hline
        $S_1$ & $S_2$ & 1 \\ \hline
        $S_2$ & $S_3$ & 2 \\ \hline
        $S_3$ & $S_4$ & 3 \\ \hline
        $S_4$ & $S_5$ & 4 \\ \hline
        $S_5$ & $S_0$ & 5 \\ \hline
    \end{tabular}
\end{center}

Next, we need to design the state assignment step:
\begin{equation*}
    S_i \mapsto i (0 \le i \le 5)
\end{equation*}

Since $2^3 = 8 > 6$, we can represent them using 6 concrete values,
namely:
\begin{align*}
    S_0 \mapsto \langle 0, 0, 0 \rangle = 000_{2}\\
    S_1 \mapsto \langle 1, 0, 0 \rangle = 001_{2} \\
    S_2 \mapsto \langle 0, 1, 0 \rangle = 010_{2} \\
    S_3 \mapsto \langle 1, 1, 0 \rangle = 011_{2} \\
    S_4 \mapsto \langle 0, 0, 1 \rangle = 100_{2} \\
    S_5 \mapsto \langle 1, 0, 1 \rangle = 101_{2} 
\end{align*}

And capture:
\begin{align*}
    &Q = \langle Q_0, Q_1, Q_2 \equiv \text{the current state} \\
    &Q' = \langle Q'_0, Q'_1, Q'_2 \rangle \equiv \text{the next state}
\end{align*}
in a 3-bit register (so like 3 latches or 3 flip-flops)

We can rewrite the abstract labels to give us some concrete truth table:
\begin{center}
    \begin{tabular}{|c c c|c c c|c c c|}
        \hline
        $Q_2$ & $Q_1$ & $Q_0$ & $Q'_2$ & $Q'_1$ & $Q'_0$ & $r_2$ & $r_1$ & $r_0$ \\ \hline
        0 & 0 & 0 & 0 & 0 & 1 & 0 & 0 & 0 \\ \hline
        0 & 0 & 1 & 0 & 1 & 0 & 0 & 0 & 1 \\ \hline
        0 & 1 & 0 & 0 & 1 & 1 & 0 & 1 & 0 \\ \hline
        0 & 1 & 1 & 1 & 0 & 0 & 0 & 1 & 1 \\ \hline
        1 & 0 & 0 & 1 & 0 & 1 & 1 & 0 & 0 \\ \hline
        1 & 0 & 1 & 0 & 0 & 0 & 1 & 0 & 1 \\ \hline
        1 & 1 & 0 & ? & ? & ? & ? & ? & ? \\ \hline
        1 & 1 & 1 & ? & ? & ? & ? & ? & ? \\ \hline
    \end{tabular}
\end{center}

The big block of `don't care' means that we should never really reach
those. We can also apply the technique of Karnaugh maps.

And that's it for the Dan Page bit of the course!

\part{Digital Logic and Computer Processors}
\label{part2}

\section{Data, control, and instructions}%
\label{sec:Data}

Abstraction is pretty great. A higher level only needs to know how to
interface with the level directly below it. So, why should we care about
hardware? Well, it's kind of important because if we write a program
that's too slow, or it isn't running right, having a grasp of how a
computer works lower down the levels of abstraction will really help us
to be able to fix whatever problems might arise.

A processor is dictated by two separate functions:
\begin{itemize}
    \item \textbf{Control}: the information and instructions
    \item \textbf{Data}: the information operated on to get the result.
\end{itemize}

These two influences form two paths into the processor logic.

The definition of \textbf{data} is simply some stored information. It can
be stored or formatted in a number of ways. Data stores information and
forms input to, and outputs from, calculations.

Data is stored in storage elements. There are many different storage
elements in modern processing systems. During this course, we only need to
focus on two: \textbf{memory} and \textbf{registers}. For this section, we
can treat them both as \textit{black boxes}, meaning that we don't really
care what's going on inside. Processor instructions always operate on data
in registers, and if they're lucky, they get to operate on data in memory,
too.

\textbf{Control} is also information, but the usage differs slightly from
data. It specifies what needs to be done and is applied to a system,
rather than consumed by it.

\subsection{Instructions}%
\label{sub:Instructions}
Instructions are really useful for telling the computer exactly what we
want to be doing. For example, if an instruction ``A" is called, then the
processor should do ``X". At a high level, we can treat them as abstract
symbols, whose value causes different processor behaviour. Instructions
need to be \textbf{decoded} before they can be used.

An instruction is \textit{also} information. However, it has a defined
purpose, which is to specify the exact amount of work that needs to be
done by a processor. This leads it to have a specific form and formatting.
Only a subset of all possible control values are actually valid
instructions.

Instructions allow the encoding of control information that is necessary
to control a computing system. The trick is to use a unique code (called
the \textbf{op-code}) to signify a unique function.

There are many different possible encodings of instructions for the same
meaning. Each system has its own tailored decode module to figure out the
meaning of the op-codes to generate control signals.

There is more than one way to make a decoder. It could be combinatorial
logic, demultiplexer based, or lookup based.

\section{Memory}%
\label{sec:Memory}

Memory, simply, is a place to store information. There are two basic
operations that we can perform on memory: \textbf{read} and
\textbf{write}. Each piece of information in memory is assigned to a
unique address. In order to access or update information, we need to
specify this address to a memory, and then our information can be returned
or changed. Addresses are specified as indexes. Values can be op-codes.

A single memory location can combine both an instruction and some data.
This can be useful for constant based operations. Consider
\lstinline|ADD1| or something that loads a constant, such as
\lstinline|MOVE2|. We could express \lstinline|ADD| as 1, therefore we
could also express \lstinline|ADD1| as \lstinline|11|. There is a memory
hierarchy that looks like this:
\begin{enumerate}
    \item Register
        \begin{itemize}
            \item 32 words
            \item Access time of $<1$ ns
        \end{itemize}
        \item L1 cache
            \begin{itemize}
                \item ~32KB
                \item ~1ns access time
                \item Part of SRAM
            \end{itemize}
        \item L2 cache
        \begin{itemize}
            \item ~512KB
            \item 5-10ns access time
            \item Part of SRAM
        \end{itemize}
        \item L3 cache
        \begin{itemize}
            \item 1-8MB
            \item 10-100ns access time
            \item Part of SRAM
        \end{itemize}
    \item Main memory (DRAM)
    \begin{itemize}
        \item 1-16GB
        \item ~100ns access time
    \end{itemize}
    \item Hard disk/SSD
        \begin{itemize}
            \item 100-1000GBs
            \item ~10ms access time
        \end{itemize}
\end{enumerate}

\subsection{Addressing}%
\label{sub:Addressing}
Addressing is when we want to access memory, we need to specify which
memory address to use. For example: \lstinline|MEM[10]| means to access
memory address 10. Ideally, we could specify a memory address directly
every time, but that's not always possible. Sometimes, we want to specify
a sequence of addresses. Therefore, we have invented a load of different
ways to specify a memory address.

\subsubsection{Immediate addressing}%
\label{ssub:immediate}
Immediate addressing is when data is supplied \textbf{in an instruction}.
There is no real memory address and all information is embedded in the
instruction. Also, data is immediately available. It's really fast and
simple (the simplest). An example looks like: \lstinline|rl <- 42|.

All of the information is embedded in instruction, so it's predictable.
This makes it really fast. It's pretty easy to understand and it's good
for optimisers to analyse. Unfortunately, in the words of the mighty Dan
Page, there's no free lunch. There is a lack of flexibility and it's gotta
be inserted statically. There's a limited range of instructions (seeing as
it's limited by the permitted number of operand bits in the opcode).

\subsubsection{Direct addressing}%
\label{ssub:direct}
An instruction like: \lstinline|MEM[10]| is pretty cool, but how is it
formed? It's formed in the kind of format: Operation | Operand1 |
Operand2. For example \lstinline|6, 10, 42|. The exact memory address used
is embedded in the instruction. This is known as \textbf{direct
addressing}. The exact memory address used is embedded in the instruction.

Direct addressing has the same pros and cons as immediate
addressing (\ref{ssub:immediate}). But, it's a little slower in return for
a larger range.

\subsubsection{Memory-indirect addressing}%
\label{ssub:memory-indirect}

Memory-indirect addressing solves the problem of limited range by storing
the address to be accessed in memory itself. 

An example would be: MEM[MEM[42]] which means to go and look at the memory
address in 42 and fetch the value. That value is the address to write the
value in r1 to.

It's good because it's got a larger range and the source memory location
for the address may be dynamically changed.

Unfortunately, there are some bad points. The first memory address is
still statically compiled. The range restriction is just changed to the
initial memory range. It's also slower than direct addressing.

\subsubsection{Register-Indirect addressing}%
\label{ssub:register-indirect}

This method provides even more flexibility. It uses the register's value
as the memory address: MEM[r1] <- r1.

There are loads of advantages to register indirect addressing like the
memory address can be dynamically computed and the value does not need to
be stored in the instruction thereby reducing code size. The register is
internal to the processor so it's faster and more energy efficient.

This also allows for native support of pointers. Accessing indirectly is
equivalent to a dereferencing operation, like *p in C.

\subsubsection{Indexed addressing}%
\label{ssub:indexed}
Sometimes, you just gotta define a base address and access memory based on
this. It's pretty useful for stacks, arrays, and caches etc. Indexed
addressing extends indirect addressing to support this. We have a
\textbf{base} and an \textit{offset}.

Normally, the base and the offset are both stored in the registers, but
this doesn't have to be the case. We get instructions like: MEM[r1 + r2]
<- r3. Here, r1 is the base and r2 is the offset. Base and offset can be
varied independently.

Many implementations support the base and offset construct natively.
Architectures often have a dedicated register to help, normally called
something like the stack pointer or the base register.

The stack/base registers may or may not be general purpose depending on
the architecture. The offset usually comes from an additional general
purpose register. An example of indexed addressing based on an array. 

\section{Memory Hierarchy}%
\label{sec:memory-hierarchy}

The ideal computer and the ideal memory have slightly different
definitions. The idea computer will have:

\begin{itemize}
    \item Zero-cycle Latency instruction execution and data supply
    \item Infinite capacity instruction execution and data supply
    \item Zero cost instruction execution and data supply
    \item Perfect Control flow
    \item Infinite bandwidth data supply
\end{itemize}

The ideal memory will have:

\begin{itemize}
    \item Zero access time
    \item Zero cost
    \item Infinite capacity
    \item Infinite bandwidth
\end{itemize}

Unfortunately, bigger memory means that it takes longer to determine the
location. Faster is also more expensive because of the memory technology.
Also, higher bandwidth is more expensive because of the increased need for
ports, banks, and higher frequency.

\subsection{DRAM}%
\label{sub:DRAM}
One bit of data is stored as a charge on the capacitor. It also has one
transistor for reading or writing. It loses charge when read and it also
loses charge over time

\subsection{SRAM}%
\label{sub:SRAM}
This consists of two cross-coupled inverters that store one bit. Six
transistors are needed for this (two for each inverter and two for the
read/write). It retains the memory if it's powered so there's no need for
a refresh.

\subsection{Memory bank (generic)}%
\label{sub:memory-bank}
A memory bank is a 2D storage array of memory cells (made up of either
SRAM or DRAM cells). To access a cell in the bank, we decode the row from
the instruction, and activate the lines. Next, we decode the column
address and select a subset of the row via the bit lines. Finally, we
amplify the row data and send this as output.

\subsection{Comparisons}%
\label{sub:SRAM vs DRAM}
Here is a table showing the differences between SRAM and DRAM:

\begin{center}
    \begin{tabular}{r|l}
        \textbf{SRAM} & \textbf{DRAM} \\ \hhline{=|=}
        Volatile & Volatile \\ 
        Faster access (no capacitor required) & Slower access (capacitor
        required) \\ 
        Lower density & Higher density \\ 
        Higher cost & Lower cost \\ 
        No need for refreshing to keep the data & Refreshing is mandatory
        \\ 
        Can be used for logic & Capacitors are bad for logic operations
    \end{tabular}
\end{center}

Here is a comparison between hard disk against flash drives

\begin{center}
    \begin{tabular}{r|l}
        \textbf{Hard disk} & \textbf{Flash drive} \\ \hhline{=|=}
        Non-volatile & Non-volatile \\
        Data stored as magnetisation patterns on circular disks & Data
        stored in floating gate transistors \\
        Mechanical (has moving parts) & No moving parts \\
        Slower & Faster \\
        Cheaper & More expensive
    \end{tabular}
\end{center}

\subsection{Locality}%
\label{sub:Locality}
Locality just means that things that are nearby are related. For example,
things that just happened give a pretty good estimate for the future.
Additionally, things that are close together are normally related (such as
people you sit close to are normally your friends).

This is the same in memory:
\begin{itemize}
    \item \textbf{Temporal locality} is a typical program that accesses
        the same memory location repeatedly for a short time.
    \item \textbf{Spatial locality} is a program that references a cluster
        of nearby addresses. The instructions tend to be in sequence and
        arrays or data structures are stored in nearby memory locations.
\end{itemize}

\subsubsection{Caching}%
\label{ssub:Caching}
Caching exploits both temporal and spatial locality. Recently accessed
data is automatically stored in a fast memory because the typical access
time is much faster (since the memory is physically closer). Similarly,
information that is stored in memory addresses that are \textit{nearby}
recently accessed addresses mean that typical access time decreases
because a program is likely to want to access the nearby memory.

\section{Instruction Set Architectures (ISAs)}%
\label{sec:ISAs}
No, an ISA is not one of those bank account. Actually, it's a contract
between the software and the hardware. More specifically, it's the
definition of the instructions that are supported by a particular
processor. The ISA specifies the exact behaviour of instructions fed to
the processor. The ISA specifies the exact behaviour of instructions fed
to the processor. The ISA should be designed sympathetically for the
programming language that will write code for it.

An ISA includes (but is not limited to):
\begin{itemize}
    \item Op-codes
    \item Data types
    \item Registers
    \item Memory addressing modes
    \item Memory architecture
    \item Interrupt and exception handling
    \item I/O
\end{itemize}

An instruction is a definition of a unique operation code as well as some
contextual information. For example, the instruction `1 01' could be
decoded with:
\begin{itemize}
    \item 1 As the op-code (move in this case)
    \item 01 as the argument. Here, we have the constant immediate value
        1.
\end{itemize}

\subsection{Decoding}%
\label{sub:Decoding}
All instructions need to be uniquely decodable. It'd be outrageous if some
strings decoded to two different things. Some ISAs don't defined all of
the possible bit-strings, which is allowed, but can cause some problems if
an undefined instruction is stumbled upon.

\subsection{Creating Instructions}%
\label{sub:Creating Instructions}
So, to create an ISA, we need to have a list of instructions and their
possible arguments. The actual purpose of the ISA will influence which
instructions make it to the final set of instructions.

Most instructions naturally fall into one of these classes:

\begin{itemize}
    \item Arithmetic
    \item Comparison
    \item Memory 
    \item Control flow
\end{itemize}

Simple processors only need those 4 classes, but some high performance
ones might employ:

\begin{itemize}
    \item Vector processing
    \item Digital signal processing
\end{itemize}

When we select the instructions to be a part of the ISA, we need to think
about the tasks in hand. There are two categories: RISC and CISC:

\subsection{Categories}%
\label{sub:Categories}

\subsubsection{RISC}%
\label{ssub:RISC}
RISC is the reduced instruction set computer. The aim is to produce a
minimal set of instructions that are fast and lightweight to execute. Each
instruction should hold a similar runtime, and complex instructions should
be made up of a few simple ones. Normally, the instructions in the RISC
model have a fixed length. RISC machines run fast in general, but they
need a lot of instructions to do a complex job.

\subsubsection{CISC}%
\label{ssub:CISC}
CISC is the complex instruction set computer. They are pretty much the
opposite of a RISC. If there is a specialised operation to be done, then
add a dedicated instruction. Because there are a lot of instructions that
are complicated, the programs are smaller. The processor is normally quite
complex and slow unfortunately, along with mixed length instructions which
leads to unpredictability and compiler nightmares. 

\subsection{Options}%
\label{sub:Options}

There are a few options for ISAs to take into account:

\subsubsection{Instruction choices}%
\label{ssub:instr-choices}
For many operations, you may have a set of choices (such as \lstinline|r0 <- r0 + r1|). How many ways can we do this?

We could have multiple accumulators ACC1 and ACC2 such that:

\begin{enumerate}
    \item \lstinline|ACC1 <- r0|
    \item \lstinline|ACC2 <- r1|
    \item \lstinline|ACC1 <- ACC1 + ACC2|
    \item \lstinline|r0 <- ACC1| 
\end{enumerate}

This only takes four very simple instructions. Another option would be to
use a stack:

\begin{enumerate}
    \item \lstinline|PUSH [r0]| 
    \item \lstinline|PUSH [r1]| 
    \item \lstinline|ADD| 
    \item \lstinline|POP [r0]| 
\end{enumerate}

Where r0 is assumed to be a memory address. We could even simplify it down
to three instructions:

\begin{enumerate}
    \item \lstinline|ACC <- r0          : MOV ACC, r0|
    \item \lstinline|ACC <- ACC + r1    : ADD r1| 
    \item \lstinline|r0 <- ACC          : MOV r0, ACC| 
\end{enumerate}

Or even two:

\begin{enumerate}
    \item \lstinline|MOV ACC, r1| 
    \item \lstinline|ADD r0, r0, ACC| 
\end{enumerate}

How can we possible choose between all of these possibilities? Well, we
kind of just need to take into account all of the tradeoffs.

Here is a comparison of the tradeoffs:

\begin{center}
    \begin{tabular}{c|c}
        \textbf{Tradeoff 1: More instructions $\rightarrow$ more bits} &
        \textbf{Tradeoff 2: More operands $\rightarrow$ more bits} \\ \hhline{=|=}
        More instructions, greater information & More bits, more work \\
        \hline
        More space, more energy & More space, more energy
    \end{tabular}
\end{center}

The choices of instruction specificity really impacts how a machine might
be programmed. It can imply a choice of paradigm, and here are some
examples:
\begin{itemize}
    \item Implicit source and destination $\rightarrow$ stack machine
    \item Implicit destination $\rightarrow$ Accumulator machine
    \item Explicit source and destinations $\rightarrow$ register machine.
\end{itemize}

Now, we need to cover the useful space of execution needs. This means that
we need no repetition, no coverage where it's not needed, Minimum size
to get the job done, check if the instruction inclusion is harmful
(Amdahl's law).

\subsection{Amdahl's law}%
\label{sub:Amdahl's law}
If a new instruction is added to speed up an operation by $s$ times, this
law gives the maximum overall speedup:

\begin{align*}
    \text{speedup } = \frac{s}{1+f(s-1)}
\end{align*}

Where $0 \le f \le 1$ is the fraction of time spent NOT running the
operation in question in the original design.

\subsection{Encoding instructions}%
\label{sub:encoding}
Now we've selected our set of instructions, we need to encode them
\textit{unambiguously}. A common way to separate them is to do it into
\textbf{op-codes} and \textbf{operand} areas. 

Space is a major factor in deciding the instruction encodings. Ultimately,
anything can be improved at the cost of it taking more space. The
instruction lengths can vary, too. Some instructions need to provide more
information than others and therefore it'll be a longer instruction. If an
instruction can be shorter, then that's good. Variable length instructions
can save space too.

Unfortunately, it's not all plain sailing to use variable length
instructions:

\begin{center}
    \begin{tabular}{l|p{65mm}}
        \textbf{Pros} & \textbf{Cons} \\ \hhline{=|=}
        Size efficient & Almost infinite amount of possibilities causing
        complex decoding \\ \hline
        Can be extended to support more operands & Pipeline slowdown \\
        \hline
        Optimal in some sense & 
    \end{tabular}
\end{center}

In comparison, fixed length instructions might be sub-optimal. They also
might limit the length of immediates. The major advantage is that it has
a simplified decode stage. It also tends to be a winner in most modern
designs.

\part{8-bit computer and Software Applications}
\section{Compilers}%
\label{sec:Compilers}
We're going to be looking at an 8 bit machine and how we'd go about
building a machine. Anytime you start with producing a computer or
processor, we need to make an instruction set. How do we design an
instruction set while keeping it simple?

\subsection{Instruction set}%
\label{sub:Instruction set}
Firstly, in any machine, we need some \textit{registers}. We need one or
two things to hold the arithmetic logic. The registers we have are:
\begin{itemize}
    \item \textbf{PC} - The program counter
    \item \textbf{AREG} - Holds the arithmetic stuff
    \item \textbf{BREG} - Same as AREG
    \item \textbf{OREG} - Holds the instruction and the operand that comes
        with the instruction
\end{itemize}

These two registers are about the minimum that we can get away with. Now,
we have two parts: the \textbf{function} and the \textbf{operand}. These
are both 4-bits long, meaning that each instruction is 8-bits long. In
terms of the instructions, there are three or four different classes of
instructions:
\begin{itemize}
    \item Load
    \begin{itemize}
        \item LDAM - load from A
        \item LDBM - Load from B
        \item STAM - Store in A
    \end{itemize}
    \item Load (constants)
    \begin{itemize}
        \item LDAC - Load A with a constant
        \item LDBC - Load B with a constant
        \item LDAP - Load address in program: This means that we can
            supply an address of the program as an operand, and store it
            in a register, and this means that we can use it as a
            callback.
    \end{itemize}
    \item Indirect loads
    \begin{itemize}
        \item LDAI - Load A indirectly
        \item LDBI - Load B indirectly
        \item STAI - Store A indirectly
    \end{itemize}
    \item Arithmetic operations and branches
    \begin{itemize}
        \item ADD - Adds
        \item SUB - subtracts
        \item BR - Branch always
        \item BRZ - Branch if zero
        \item BRN - Branch if negative
        \item BRB - Branch to the contents of the B register
    \end{itemize}
\end{itemize}

We now have 15 instructions, so we have one more instruction: PFIX, which
is a prefix. This means that we are able to encode a 16-bit value by first
supplying the PFIX with the first half, and then the function with the
second half. It attaches the operand to the operand of the next
instructions. Even this instruction set is able to do some quite
complicated things, such as actually be a compiler.

The arithmetic units will be used for both doing the arithmetic and for
computing the addresses. For example, it will have to add the PC to the
operand etc.

Let's start with some registers. So, we have the registers that are
attached to the arithmetic unit via some multiplexers. We also might need
one of the values to address the memory. If we're reading data from the
memory, then we need another multiplexer to be able to take values from
either the memory or the ALU. We also need to be able to write to the
memory, but because we've set up the instructions so that we can only
write to memory from the A register, we only need to connect it to the
AREG.

The rest of the machine can be seen online, so I won't be documenting it
here because it doesn't really make sense to be described in words.

\section{Hex Architecture}%
\label{sec:Hex Architecture}
Having defined the instruction set above, we need to work out what we
actually want to do with it. So what we have is some block of memory. In
the case of our \textit{simple machine}, we can split up the memory,
working from the bottom, into:
\begin{itemize}
    \item Jump (that jumps to the beginning of the program)
    \item The stack pointer. This initially points to the top of the
        stack. As we call the functions, we then \textit{decrement} the
        stack pointer, so that we can get back to the place we came from.
        We can keep doing this with no problems.
    \item Global variables
    \item Constants
    \item Strings
    \item Program
    \item Stack
\end{itemize}

\subsection{The stack}%
\label{sub:the-stack}
As we lay out the information in the stack, when we do a procedure call,
we execute some instructions that load the return address into the AREG.
This means that the stack now holds the address that is stored in the
AREG, and then we decrement the stack pointer.

We might also want to pass some parameters to the function, and these are
also stored in the stack (right above the return state value). We can
still use the same instructions to access the incoming parameters because
we know the offset.

Let's look at the following example:
\begin{lstlisting}[]
LDBM 1
STAI 0 'bottom location of the stack
LDAC -5 'The offset required for the locals/variables etc.
OPR ADD
STAM 1
LDAM 1
LDAI 6
BRN L207
BR L206
\end{lstlisting}

So, we've jumped to this function. Now, we load the stack pointer that is
in memory location 0. We then load -5, which is the space required for
the local variables of the function. We then ADD the two together, and
then place the result back into the stack at memory location 1. Now, we're
loading up one of the incoming parameters, and then testing to see if
the result is negative or not.

The actual code of the stack use is that we use:
\begin{lstlisting}[]
LDAP
BR F
R:

F:
\end{lstlisting}

LDAP is used because it stores the program location into register A. Next,
it branches to some function F, with the return value stored in the A
register. After this, we just decrement the stack pointer in the same way
discussed above. This can be done recursively, and then we can just read
back to get the same order as before.

We have a couple of things that we need to be introduced to. There is
something called a \textbf{lexical analyser}, which takes all of the
incoming symbols and converts them into a signal, and something called a
\textbf{syntax analyser}  which converts the symbols into a kind of data
structure. After it's made into a tree (the kind of data structure that
the compiler prefers), it looks to simplify it first, and then it solves
it. For example, in the code:

\begin{lstlisting}[]
if c then p else q
\end{lstlisting}

We end up with a data structure that looks like:

\begin{lstlisting}[]
| if | c | p | q |
\end{lstlisting}

\section{The general compiler actions}%
\label{sec:compiler}
When we translate an if statement, we don't know how long the condition is
going to be, so we kind of do a rough input for the first pass, and then
an algorithm goes over the code afterwards to input the necessary offsets
and such.

When a name (of a variable) is read in, it's looked up in a name table,
which is essentially a hash table. The very first time it's read in, it's
input into the name table, but every subsequent time, it is just pointed
to the memory address associated with the name. We also have to look up
all of the things like `if', `then', `else' etc. in the same way, so we
have to hard code those in.

On a reading of an `if' keyword, we have to skip over the symbol so that
we're clear of the keyword, and then we start reading the condition. On
top of that, we look for a `then' (recursively). If we don't find that,
there's an error. At the end of the whole thing, we return a data
structure as discussed before of the `if'.

If we have a global variable AND a local variable defined as the same
name, we obviously want the compiler to only mess with the local one if
it's in the local function. This is part of the \textbf{scope rule}. After
this, we need to try and \textit{optimise} the code, such as if we see `0
+ x', we could just make it `x'. Similarly, if we see `3+5', we can just
convert this into `8'.

\subsection{The scope rule}%
\label{sub:scope-rule}
When we declare a name, we define a point in the stack that has both the
name, and the address in memory. This amount of memory is known as the
\textbf{offset}. The part of the compiler that deals with the declaration
of local variables is fairly self explanatory for the parts discussed
previously. When we encounter a name in the middle of the expression, we
go down the stack, and either find the name and return the offset of the
stack with the information of the name, or we don't find it and return an
undefined error.

\subsection{Optimisation}%
\label{sub:Optimisation}
The optimisation process does more than just optimise. It also converts
all of the relational operators into simply an equals to or a less than
comparator (because the computer is only able to compute less than or
equals operations). In the case of finding a return node, it will replace
it with an optimised return node. If we encounter a name, we see if we can
find a value, and then just return the value as the result. Additionally,
we see if it was a constant, or a computed value, so then we can convert
the tree node into a value. What we're doing here is \textit{pruning} the
tree node.

If we find an `=' symbol, then we can do a few things:
\begin{itemize}
    \item If we have `!x = !y', then we can just replace this with `x = y'
    \item If we have a `!=', then not only look ahead to see if they are
        both notted (as before), we convert the node into an equals node,
        and then insert another node to not that, so that we can just deal
        with equals signs.
\end{itemize}

If we find a `greater than' symbol, we switch around the code, so that we
always get the `less than' symbol. There are a lot more things that we can
do, such as using DeMorgan's law to simplify things, remove the 0 operands
in addition or ORs etc.

Having all these sorted out, we're now in a position where we can start to
translate the code into machine code that is actually executable.

\subsection{Translation into executable}%
\label{sub:executable}

\subsubsection{Control structure}%
\label{ssub:control}

The steps required to translate the code into executable operates much in
the same way as that of the optimiser; often recursively calling itself
and so on. It has parameters that keeps track of what to do when the
statement is finished. For example, if we have a piece of code that says:
\begin{lstlisting}[]
if (c1) then
    while c2 do p
else
    q
end if
\end{lstlisting}

This would look like this in machine code:

\begin{lstlisting}[]
c1
BRZ ELSE
START:
c2
BRZ END
P
BR START

ELSE:
    q
END:
\end{lstlisting}

There's a function called `generate statement', that has a few flags:
\begin{itemize}
    \item One that has the incoming part of the tree
    \item A flag that tells you whether once the statement is compiled, if
        it flows straight through, or output some control
    \item A flat that tells you where the control needs to go (if there is
        one)
\end{itemize}

It compiles itself recursively if it is given a semi-colon list. The
interesting part of this is that there could be something where the else
part is empty, in which case the compilation is simplified. We generate
some labels for the `if' and the `else' parts, and then compile the parts
of code.

For the `while' code, we set the label for the beginning of the loop, set
up the condition, and then the code, with a branch back to the start. This
is how control structures are generated.

Let's look at another example:

\begin{lstlisting}[]
if (x AND y) then
    p
else
    q
end if
\end{lstlisting}

We'd get:
\begin{lstlisting}[]
x
BRZ ELSE
Y
BRZ ELSE
p
BR END

ELSE:
q
END:
\end{lstlisting}

We don't even need to do any actual logic, because we just check each
operand in turn, and then branch to the else if either of them are 0.

For the less than function, we just check if one of the parameters is
zero. If not, then subtract one from the other and branch if negative.

The function calls are a little complicated. If we have the code:

\begin{lstlisting}[]
P(x,y,z)
\end{lstlisting}

The problem is that the parameters might be quite complicated. For
example, there could be procedure calls embedded in the arguments, so we
have to go digging around trying to find the code. The compiler has some
counters to this, such as to look inside the parameter list to find any
embedded functions and call those first, before generating the rest of the
instructions.

\subsubsection{Assigning variables}%
\label{ssub:Assigning variables}

When we do a simpler assignment (such as name = othername), we have a left
hand side and a right hand side, we just check the value of one, and then
assign it to the other one.

If we have a more complicated one, such as a subscripted (a[i] for
example) assignation, then we have a \textbf{base pointer} and an
\textbf{offset}. We check to see whether the offset is a value, in which
case we just generate an expression for that. However, we might be using
an index to iterate through an array. In this case, we need to generate
some code to form the address, and then use the address to store the
value. Because we have so few registers in the machine, we need some code
that will get the base, and the offset, and then store the data in the
stack. After this, we have some code to store it in the appropriate place.

If we ever run out of registers, the compiler just slaps them on the
stack.

\subsubsection{Translating expressions}%
\label{ssub:expressoins}
This final step just handles the logical operations, the function calls
and other such things like that.

\subsubsection{Code buffer}%
\label{ssub:buffer}
As seen before, when we have an `if' statement, we're going to have a
`branch zero' to some else tag, and then do the statement. But, because we
don't know how many lines of code we need to process before placing the
`else' tag, we need to put the label into the buffer. The buffer then
fixes up the offset. The offset might be longer than the 4 bits that we
get for the buffer, so we might also need a prefix. Jesus, who does this?

\section{Changing instruction sets}%
\label{sec:changing-sets}
\textbf{Bootstrapping} is where you write a language in its own language.
It originated in something like 1960. When we move a language from one
instruction set to another, we need something called \textit{ocode}.

When we write a compiler, we write them in a suitable source. So, we have
a compiler that is written in language x, that compiles in language y.
Now, we will also need something that compiles language y and so on. But,
with bootstrapping, the source and the executable are the same language,
so things are good and alright.

At some point, we might want to stop writing our source in a language, so
we write a new compiler, and then we get a new language, and eventually,
we can push out the old language entirely, and this is how you make a new
language and it's compiler in bootstrapping form. If we didn't have C,
this would be really hard, because we'd have to go right down to machine
code, or a macro generator (one that kind of brute forces it). Notice that
because we're going to throw away the old language, it doesn't matter how
inefficient it is, as long as we get to the right destination.

The process of adding new features is pretty much the same: we create a
`new' language, which is just the old language with added bells and
whistles, and then apply that to the compiler, and then away we go.

Let's explore an example of this. Say we wanted to add another `special
character' (the likes of `\textbackslash n') then we make a compiler that
recognises the new one, and then use that in recognising the new code
itself.

If we wanted to add a new key word (for example `until'), then we need to
declare the new keyword, and then add a line in the list of ones that
recognises the key word, and then pretty much just copy the while code,
but have a NOT at the end of the condition.

\subsection{Optimising a language}%
\label{sub:optimising-bootstrapping}
If we want to add some optimisations to the language using bootstrapping,
then we need to do the following.

\begin{itemize}
    \item We have the source code $S_i$ and an executable $E_i$. We
        compile the source with the executable to produce the same thing:
        $E_i = E_i(S_i)$
    \item We want to produce the optimised source $(S_{i+1})$, and this
        will be compiled with the same executable: $E_i(S_{i+1})$, this
        produces something else that won't be the same as either, so we'll
        call it $O_i$ because it can optimise. However, it hasn't had the
        optimisations applied to it.
    \item We now need to apply the optimisations to the compiler itself:
        $O_i(S_{i+1}) = E_{i+1}$.
    \item Now, we're left with $E_{i+1}(S_{i+1}) = E_{i+1}$
\end{itemize}

Sometimes, the object code is smaller, even though the source code is
bigger because of the optimisations. Conversely, it might just make the
code a bit worse, and you just won't know until you've tried it.

\section{Garbage collection}%
\label{sec:garbage-collection}

Garbage collection (and memory allocation) can get pretty hard to
understand, and therefore we have some techniques to automate this. The
techniques aren't new; they've been around for ages. It was originally
made for some random language, but now has been adopted for a few other
ones.

We have something called a \textit{heap} with some globals underneath it,
then the stack, and then the program and some constants. What we want to
do is automatically allocate some memory in the heap when the program
needs it, and then deallocate it when the program is done with it.

We might write something like \lstinline|x = [1,2,5]|, and so the garbage
algorithm will now have some objects:

\begin{enumerate}
    \item Size
    \item Flag (used to mark the elements in the heap that are
        needed)
    \item Address
\end{enumerate}

The next step is to identify which parts of the heap are needed. What we
want to be able to do is to go through the region and mark all the objects
in the heap that are pointed to by the items in the program (by things in
the stack really). In other words, we want to know which items in the
stack are pointing to elements in the heap.

There are a few options for doing this. The compiler might leave a map for
the variables that are used, but sometimes, the language doesn't do this.
Another method is just to guess. Finally, we could just explicitly mark
the data. Hardware designers would be really cross with you for this
because they'd need 65-bit designs now.

What we're going to do is steal a bit from the words in memory.
Unfortunately, you lose some functionality. Sometimes, the bottom bits
aren't even used in memory addressing, which sounds kind of weird.  When
we do this, it means that every time we add up, we need to subtract 1 from
the result because the final bit is used for the flag, but this means that
we need to change the compilation.

Now, we step through the stack and see which items are pointing to the
heap. When we find one, we set the flag in the address. Not only this, but
we need to set those items that the heap memory points to (multiple
pointers and whatnot).

To make space available, we have a pointer in the heap, that goes from
bottom to top. To avoid fragmenting it, we move the objects down so that
they're \textbf{compacted}. BUT, if we just do this willy nilly, then
we'll run into errors because all of the pointers will be messed up and
there's no easy way to find out what items are pointing to a given entity.
So, there's another step that we need to consider.

\subsection{Compacting}%
\label{sub:Compacting}
To move the items down, we need to consider the items that are pointing to
a given entity. To do this, we start at the bottom of the stack. If an
item is not \textit{marked} (i.e. doesn't has a flag), then we skip over
it. We know how much to skip by because the size is at the bottom of the
entity. If it is marked, then we store the address that it will be moved to
at the bottom. We're allowed to do this because there's a spare word in
each data item. When we find another item, we do the same; recording the
final position of each item, until each marked item has a destination
recorded

Having done this, we can return to scanning the stack. When we step
through the stack and identify the pointers to the heap, we can now update
the pointers in the stack to the `final address' information that we
recorded before. We have to carry on this process through into the
globals, and also the heap itself.

Finally, we can go back once more to the stack, and perform the copies.
The items that are not marked, are replaced, and the old marked locations
are unmarked, ready to be overwritten.

This is a fairly simple collector, but there are simpler ones because this
one can deal with arbitrary sized data.

\subsection{Marking}%
\label{sub:Marked pointers?}
We're not done with this yet, however. When we mark an object in the heap,
we then also need to mark the items that the new marked item is pointing
to (else it will be erased). The most obvious one is to recursively mark
the items. This is not a good solution to the problem, unfortunately. This
is because to do something recursively, we need a stack, which uses some
memory. The issue with this is that we need to allocate a relatively large
amount of space for this stack.

An alternative position is to iteratively mark the items (go to the items
that the thing points to and mark it. Rinse and repeat). This would take
so many iterations, which means that the performance take a hit, rather
than the size (as with the recursive method).

Thankfully, there is a solution that isn't huge or slow. The way that
we're going to do it is to implement some more pointers: one to
\textbf{current} and one to \textbf{next} and one to \textbf{previous} .
When we want to move down a level, we update the \textit{current} pointer
to the new item, the previous to the current, and the next to the
following item.

Most of the time, we do a combination of the above (such as using
iteration to a certain depth, and then using the third one). This third
one is called the `stop the world' algorithm, because when it runs out of
memory, it stops. 

Garbage collection is really hard because there are lots of points in
which things can go wrong, such as the program trying to access the data
as the collector is moving the data. This can be solved using locks and so
on, but you get the point: garbage collection is real hard. 

\section{Communication, Interrupts, and Protection}%
\label{sec:interrupts}
We need some means by which two independent devices can communicate. So
far, we've looked at the two phase clock. However, when we start
communicating between computers, they each have their own clocks, so we
need some kind of encoding standard so that the data can be read. The
traditional method is the `handshake'. This means that computer A sends
out a request, computer B confirms the request, computer A confirms the
confirmation, and then data transfer can continue.

In layman's terms, the sender puts the request on the data wire. We
assert the request to confirm that the data is actually valid. Now, the
sender waits until the receiver puts the acknowledge wire on. The sender
can then take off the request from the data wire, and then the wire can
change state.

There are a few optimisations that can be done to this, but this is called
the `return to 0' scheme. A faster way of doing this is to do it on the
edges of the changes, to make it faster. It almost halves the data
transfer rate, but it's still not perfect because it's impossible to
guarantee the transmission speed of the wire, because the receiver might
see the request at a time when the data becomes invalid.

An alternative to this is that we have three signals: REQ0, REQ1, and ACK.
If we want to request a 0, we make REQ0 1, and wait for the ACK to be 1.
Next, we might want to request a 1, so we make REQ1 1, and wait for ACK to
change state (remember that it was 1 so now we wait for it to become 0).
We might then want another 0, so we change the REQ0 to 0 once more, and
wait for the ACK to change state. This repeats. This is a good method
because it doesn't matter how long it takes for the signal to reach
because we are only waiting for the changes in the signals.

Did you want another method? No?? Tough. This one has four wires for
transmission. Now we have 2 bits of transmission. You could EVEN have 6
wires, and once 3 change state, the recipient responds. This means we can
encode 20 values. These two aren't hugely used, but some psychopaths do
use it.

\subsection{Transferring data}%
\label{sub:Transferring data}
When we communicate between devices, we need a way to transfer the data.
When a device receives some information, it has to take it into some flip
flop to be able to understand the data. Unfortunately, the flip flop COULD
change state at the exact same time as the clock signal, which would make
it enter the meta-stable state. As long as you wait a while, the flip flop
will stabilise, but if the circuit isn't designed appropriately, then the
computer will fail (a mistake that has actually been made in the past).

Now we know how it works physically, we need to know how to code these. We
could of course have another register (or a special memory location) that
behaves just like any other memory location, but the memory is solely
reserved for data transfer. Normally, we have a system to allow the
program to transmit data (or wait to communicate) while the program is
still going (asynchronously), and this is usually achieved through
interrupts.

\subsection{Interrupts}%
\label{sub:Interrupts}
The basic idea is that the program tries to communicate some data, and
then the interrupt occurs once the transfer has completed, meaning that
the program is able to continue supplying data.

In order to make an interrupt system work, we need to transfer control on
some specific address. This is kind of like a subroutine call, but it can
happen at any point in time. We also need to restore the position we were
at. So, unlike a subroutine call, we don't always have the location of the
program, so we need to save the state automatically.

Because of this, we have another register (where the PC is transferred)
before the control is transferred to the desired device. On top of this,
the address for the control to be handled to is stored at the bottom of
memory and they are called `\textbf{interrupt vectors}'.

We have another thing called an `in kernel' flag. The \textit{kernel} is
the innermost part of the operating system is called the \textbf{kernel},
and the flag denotes whether or not we are in the kernel (or handling an
interrupt in other words).

If we are trying to transfer some data to the buffer, and then we get
another interrupt to get some data from the buffer, we might lose some
data. We therefore need some safeguard to stop the interrupts getting data
from the same buffer once the interrupt is received.

An optimisation of this can be used and this is quite important. If we
have some memory system, and some processor able to access the memory (as
well as some I/O), then when we transfer data from the I/O, the process is
very similar to what we discussed above. Somewhere in memory space, we
want to create a situation where the user cannot corrupt the memory. We
have some system that protects the program from overwriting the kernel (so
that we have something left if a program crashes). The kernel also does
some scheduling (like the OS does). 

\end{document}
