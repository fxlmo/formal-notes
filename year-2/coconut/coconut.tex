%%=====================================================================================
%%
%%       Filename:  coconut.tex
%%
%%    Description:  Coconut lecture notes
%%
%%        Version:  1.0
%%        Created:  14/11/19
%%       Revision:  none
%%
%%         Author:  YOUR NAME (), 
%%   Organization:  
%%      Copyright:  Copyright (c) 2019, YOUR NAME
%%
%%          Notes:  
%%
%%=====================================================================================

% Preamble {{{
\documentclass[11pt,a4paper,titlepage,dvipsnames,cmyk]{scrartcl}
\usepackage[english]{babel}
\typearea{12}
% }}}

% Set indentation and line skip for paragraph {{{
\setlength{\parindent}{0em}
\setlength{\parskip}{1em}
\usepackage[margin=2cm]{geometry}
\addtolength{\textheight}{-1in}
\setlength{\headsep}{.5in}
% }}}

\usepackage{hhline} 
\usepackage{mathtools} 
\usepackage[T1]{fontenc}

% Headers setup {{{
\usepackage{fancyhdr}
\pagestyle{fancy}
\lhead{Coconut: Lecture notes}
\rhead{Josh Felmeden}
\usepackage{hyperref} 
% }}}

% Listings {{{
\usepackage[]{listings,xcolor} 
\lstset
{
    breaklines=true,
    tabsize=3,
    showstringspaces=false
}

\definecolor{lstgrey}{rgb}{0.05,0.05,0.05}
\usepackage{listings}
\makeatletter
\lstset{language=[Visual]Basic,
    backgroundcolor=\color{lstgrey},
    frame=single,
    xleftmargin=0.7cm,
    frame=tlbr, framesep=0.2cm, framerule=0pt,
    basicstyle=\lst@ifdisplaystyle\color{white}\footnotesize\ttfamily\else\color{black}\footnotesize\ttfamily\fi,
    captionpos=b,
    tabsize=2,
    keywordstyle=\color{Magenta}\bfseries,
    identifierstyle=\color{Cyan},
    stringstyle=\color{Yellow},
    commentstyle=\color{Gray}\itshape
}
\makeatother
\renewcommand{\familydefault}{\sfdefault}
% }}}


% Other packages {{{
\usepackage{needspace}
\usepackage{tcolorbox}
\usepackage{soul}
\usepackage{babel,dejavu,helvet} 
\usepackage{amsmath} 
\usepackage{booktabs} 
\usepackage{tcolorbox} 
\usepackage[symbol]{footmisc} 
\renewcommand{\thefootnote}{\fnsymbol{footnote}}
\renewcommand{\familydefault}{\sfdefault}
% }}}

% tcolorbox {{{
\newtcolorbox{blue}[3][] {
    colframe = #2!25,
    colback = #2!10,
    #1,
}

\newtcolorbox{titlebox}[3][] {
    colframe = #2!25,
    colback = #2!10,
    coltitle = #2!20!black,
    title = {#3},
    fonttitle=\bfseries
    #1,
}
% }}}

% Title {{{
\title{Coconut: Lecture notes}
\author{Josh Felmeden}
% }}}

\begin{document}

\maketitle
\tableofcontents

\newpage
\section{Hamming Codes}%
\label{sec:hamming-code}
For any $m \ge 1$, the generalised Hamming code encodes up to $m-1$ data
bits with $m$ parity bits for a total of $2^m-1$ respectively bits.
\begin{itemize}
    \item The minimum distance $d=3$ (respectively 4 for SECDED) for any
        value of $m$
    \item The $i$-th bit is a parity bit if $i$ is a power of 2, otherwise
        a data bit.
    \item The parity bits are set such that the sum of all bits whose
        index $i$ has a 1 at the $j$-th position when written out in
        binary equals 0.
    \item For the SECDED version, an additional initial bit 0 is set such
        that the sum of all the bits in the codeword is zero.
\end{itemize}

The parity bits in the generalized scheme is computed such that all of the
1s in the rows must sum to zero. For example:

\begin{center}
    \begin{tabular}{c|c c c c c c c}
        0 & 1 & 2 & 3 & 4 & 5 & 6 & 7 \\
        \hline
        0 & 0 & 0 & 0 & 1 & 1 & 1 & 1 \\
        0 & 0 & 1 & 1 & 0 & 0 & 1 & 1 \\
        0 & 1 & 0 & 1 & 0 & 1 & 0 & 1 \\
        \hline
        Bit name & $P_1$& $P_2$& $m_3$& $P_4$& $m_5$& $m_6$& $m_7$
    \end{tabular}
\end{center}

So, ${\color{red}{P_4}}+ m_5 + m_6 + m_7 = 0$ where
{\color{red} $P_4$} is a parity bit, and the rest are data bits.

We use the following rules:
\begin{enumerate}
    \item ${\color{red}P_4} + m_5 + m_6 + m_7 = 0$
    \item ${\color{red}P_2} + m_3 + m_6 + m_7 = 0$
    \item ${\color{red}P_1} + m_3 + m_5 + m_7 = 0$
\end{enumerate}

Let's revisit the table. If we have the sequence:
\begin{center}
    \begin{tabular}{c|c c c c c c c}
        0 & 1 & 2 & 3 & 4 & 5 & 6 & 7 \\
        \hline
        Ex 1 & 0 & 1 & 1 & 0 & 0 & 0 & 1 \\
        \hline
        Bit name & $P_1$& $P_2$& $m_3$& $P_4$& $m_5$& $m_6$& $m_7$
    \end{tabular}
\end{center}

Then, rules \textbf{1} and \textbf{2} are violated, but \textbf{3} is not.
So, with the GHC decoding rule, we sum the indexes of violating parity
bits. Since we know that ${\color{red}P_1}$s bits are okay, but
${\color{red}P_2, P_4}$ are not, we can therefore tell that $m_6$ is the
issue, so if we flip this, then we'll probably be okay.

\subsection{Reed-Solomon codes}%
\label{sub:reed-solomon}
The idea behind these boys are that we have something called the singleton
bound. Now, if we have an encoding function $E:\sigma^k \rightarrow
\sigma^n$, the singleton $d$ is going to be $d \le n-k+1$. So this is the
singleton bound. A code with $d = n - k + 1$ is called the MDS (maximum
distance separable).

The \textbf{Reed-Solomon} code is:
\begin{itemize}
    \item Pick some points.
    \item Find the polynomial.
    \item Evaluate as much as you need it to be.
\end{itemize}

If we have a message that's not all 0, we are evaluating at $n$ different
values. Also, over a field, at most $k-1$ values are zeroes. Therefore, at
least $n-(k-1)$ points in a non zero code word are nonzero.

% Lecture from 26/11/19
\section{Information Theory}%
\label{sec:information-theory}
Firstly, let's lay some ground rules, shall we?

\begin{itemize}
    \item $\Sigma$ = Alphabet (finite, non empty set)
    \item $X =$ Alphabet (with only 0s and 1s)
    \item $\Sigma^+ =$ Non empty words over $\Sigma$
    \item $\Sigma^* =$ Words over $\Sigma$ including $\epsilon$ (empty)
\end{itemize}

We want to be able to send some message $\underline m \in \Sigma^+$ to some
channel with some noise.

Here are some more definitions:
\begin{itemize}
    \item $f : X \rightarrow y$ is injective if $x \not = x' \rightarrow
        f(x) \not = f(x)$ and $f(x) = f(x) \rightarrow x = x'$.
    \item $f : X \rightarrow y$ is surjective if every $y \le y$ has some
        $x \in X$ such that $f(x) = y$.
    \item If both are true, then it is bijective.
\end{itemize}

Finally, $\text{Image}(f) = \{y \in Y | y = f(x) \text{ for some } x \in
X\}$.

Now, let's look at an example:
\begin{center}
    \begin{tabular}{c|c c c c}
        $\sigma$ & A & B & C & D \\ \hline
        $f(\sigma)$ & $0$ & $1$ & $00$ & $01$
    \end{tabular}
\end{center}

This is a problem because does $00 =$ c or AA? Instead, we could do
something like:

\begin{center}
    \begin{tabular}{c|c c c}
        $\sigma$ & A & B & C \\ \hline
        $f_1(\sigma)$ & 1 & 10 & 100
    \end{tabular}
    \quad or \quad
    \begin{tabular}{c|c c c}
        $\sigma$ & A & B & C \\ \hline
        $f_2(\sigma)$ & 1 & 01 & 001
    \end{tabular}
\end{center}

Now, $f_2$ is actually better because we know when something ends by the
$1$. The first one leaves us waiting for additional $0$s. We call the
second function \textbf{prefix-free}.

\begin{titlebox}{blue}{Prefix free}
An encoding $E$ is prefix free if for all $\underline m, \underline m'$,
$E(\underline m)$ is not a prefix of $E(\underline m')$
\end{titlebox}

\textbf{Theorem}: If $E$ is injective and prefix-free, then $E^+$ is
injective.

Let's now assume that $\underline m \not = \underline m'$ but
$E^+(\underline m) = E^+(\underline m')$. We can now work backwards and
prove that this is a contradiction:
\begin{itemize}
    \item Assume $|\underline m| \le |\underline m'| (w\log)$
    \item If $\underline m$ is prefix of $\underline m'$, then $\exists
        \underline s : \underline m' - \underline{ms}$
    \item $E(\underline m') = E^+(\underline m)E^+(\underline s) =
        E^+(\underline m) \rightarrow E^+(\underline s) = \epsilon$
        \begin{itemize}
            \item This is a contradiction, because we said that they
                should evaluate to the same thing before but now one is
                empty.
        \end{itemize}
\end{itemize}



\end{document}
