%%=====================================================================================
%%
%%       Filename:  lang-eng.tex
%%
%%    Description:  formal notes for lang-eng
%%
%%        Version:  1.0
%%        Created:  14/10/19
%%       Revision:  none
%%
%%         Author:  Josh Felmeden (), nk18044@bristol.ac.uk
%%   Organization:  
%%      Copyright:  Copyright (c) 2019, Josh Felmeden
%%
%%          Notes:  
%%
%%=====================================================================================

% Preamble {{{
\documentclass[11pt,a4paper,titlepage,dvipsnames,cmyk]{scrartcl}
\usepackage[english]{babel}
\typearea{12}
% }}}

% Set indentation and line skip for paragraph {{{
\setlength{\parindent}{0em}
\setlength{\parskip}{1em}
\usepackage[margin=2cm]{geometry}
\addtolength{\textheight}{-1in}
\setlength{\headsep}{.5in}
% }}}

\usepackage{hhline} 
\usepackage{mathtools} 
\usepackage[T1]{fontenc}

% Headers setup {{{
\usepackage{fancyhdr}
\pagestyle{fancy}
\lhead{Language Engineering - A nice set of notes}
\rhead{Josh Felmeden}
\usepackage{hyperref} 
% }}}

% Listings {{{
\usepackage[]{listings,xcolor} 
\lstset
{
    breaklines=true,
    tabsize=3,
    showstringspaces=false
}

\definecolor{lstgrey}{rgb}{0.05,0.05,0.05}
\usepackage{listings}
\makeatletter
\lstset{language=[Visual]Basic,
    backgroundcolor=\color{lstgrey},
    frame=single,
    xleftmargin=0.7cm,
    frame=tlbr, framesep=0.2cm, framerule=0pt,
    basicstyle=\lst@ifdisplaystyle\color{white}\footnotesize\ttfamily\else\color{black}\footnotesize\ttfamily\fi,
    captionpos=b,
    tabsize=2,
    keywordstyle=\color{Magenta}\bfseries,
    identifierstyle=\color{Cyan},
    stringstyle=\color{Yellow},
    commentstyle=\color{Gray}\itshape
}
\makeatother
\renewcommand{\familydefault}{\sfdefault}
% }}}


% Other packages {{{
\usepackage{needspace}
\usepackage{tcolorbox}
\usepackage{soul}
\usepackage{babel,dejavu,helvet} 
\usepackage{amsmath} 
\usepackage{booktabs} 
\usepackage{tcolorbox} 
\usepackage[symbol]{footmisc} 
\renewcommand{\thefootnote}{\fnsymbol{footnote}}
\renewcommand{\familydefault}{\sfdefault}
% }}}

% Title {{{
\title{Language Engineering - A nice set of notes}
\author{Josh Felmeden}
% }}}

\begin{document}

\maketitle
\tableofcontents

\newpage

\section{Introduction to Semantics}%
\label{sec:intro-semantics}

Semantics are really complex and they actually exist in the real world as
problems that can arise when the semantics are unclear. In the example of
the Derek Bentley case, Bentley tells Chris (who is holding a gun, and a
policeman standing in front of him to `let him have it!'. Here, it
appears that he could be talking about the gun, or to kill him. The same
kind of thing can happen in computing when we are unsure of the references
of certain objects. 

Here are some examples learned from natural languages:
\begin{itemize}
    \item Syntactic complexity
        \begin{itemize}
            \item Jack built the house the malt the rat the cat killed ate
                lay in
        \end{itemize}
    \item Syntactic ambiguity
        \begin{itemize}
            \item Let him have it, Chris!
        \end{itemize}
        \item Semantic Complexity
            \begin{itemize}
                \item It depends on what the meaning of the word `is' is!
            \end{itemize}
        \item Semantic ambiguity
            \begin{itemize}
                \item I haven't slept for ten days
            \end{itemize}
            \item Semantic undefinedness
            \begin{itemize}
                \item Colourless green ideas sleep furiously
            \end{itemize}
            \item Interaction of syntax and semantics
                \begin{itemize}
                    \item Time flies like an arrow, fruit flies like a
                        banana.
                \end{itemize}
\end{itemize}

We can apply these things to computing terms, too.
\begin{itemize}
    \item Syntactic complexity
            \begin{lstlisting}
x-=y = (x=x+y) - y      //switches variables x and y
            \end{lstlisting}
    \item Syntactic ambiguity
\begin{lstlisting}
if (...) if (...) ..; else ..       //dangling else
\end{lstlisting}
        \item Semantic Complexity
\begin{lstlisting}
y = x++ + x++       //sequence points
\end{lstlisting}
        \item Semantic ambiguity
\begin{lstlisting}
(x%2=1) ? "odd" : "even"        //unspecified in C89 if x<0
\end{lstlisting}
            \item Semantic undefinedness
\begin{lstlisting}
while(x/x)      //division error or infinite loop
\end{lstlisting}
            \item Interaction of syntax and semantics
\begin{lstlisting}
A * B       //lever hack
\end{lstlisting}

\end{itemize}

To put this another way:
\begin{itemize}
    \item \textbf{Syntax}: concerned with the form of expressions and
        whether or not the program actually \textit{compiles}
    \item \textbf{Semantics}: concerned with the meaning of expressions
        and what the program does when it \textit{runs}
    \item \textbf{Pragmatics}: concerned with issues like design patterns,
        program style, industry standards, etc.
\end{itemize}




\end{document}
