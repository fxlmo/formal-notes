%%=====================================================================================
%%
%%       Filename:  security-101.tex
%%
%%    Description:  Notes about security that are so nice that I should be
%%                  knighted
%%
%%        Version:  1.0
%%        Created:  15/01/19
%%
%%         Author:  Josh Felmeden (), nk18044@bristol.ac.uk
%%
%%=====================================================================================

% Preamble
\documentclass[11pt,a4paper,titlepage]{scrartcl}
\usepackage[english]{babel}
\typearea{12}

% Set indentation and line skip for paragraph
\setlength{\parindent}{0em}
\setlength{\parskip}{1em}
\usepackage[margin=2cm]{geometry}
\addtolength{\textheight}{-1in}
\setlength{\headsep}{.5in}

% Headers setup
\usepackage{fancyhdr}
\pagestyle{fancy}
\lhead{Security-101: Notes that would make my mum proud}
\rhead{Josh Felmeden}

% Title
\author{Josh Felmeden}
\date{January \\ 2019}
\title{Security-101: Notes that would make my mum proud}

\usepackage{amsmath} 
\usepackage{booktabs} 
\usepackage[symbol]{footmisc}
\renewcommand{\thefootnote}{\fnsymbol{footnote}}

\begin{document}
\pagenumbering{roman}
\maketitle
\tableofcontents 
\newpage
\pagenumbering{arabic}

\section{Encoding}%
\label{sec:Encoding}
The first thing to note is that encoding is NOT encryption. They're
actually very different. Encoding is used for safe transfer of data. Say
you want to send a message by form of an e-mail, or telegraph etc. E-mails
are all encoded by the e-mail program, so you don't need to worry about
it. Morse code is an example of encoding, but all of the characters have
different lengths.

In modern terms, a \textbf{bit} can be either 0 or 1 (obviously, please
tell me you knew that). Baudot code is kinda similar to Morse code, except
that each character has a fixed length of 5 holes, that can be either
covered or not. This is like bits, because if the hole is covered, we can
represent this as a 1, and the uncovered holes are represented by a 0.

The formal definitions are:
\begin{itemize}
    \item \textbf{Encoding} is a publicly known way to transform a message
        so that you can send it over a particular channel. There will be
        no secrecy from this so anyone can decode it.
    \item \textbf{Encryption} is transforming a message so that only
        the intended recipients can decrypt it. It requires a key.
\end{itemize}

A way to encode is a code, while the way to encrypt is a cipher.

\subsection{ASCII}%
\label{sub:ASCII}

ASCII was introduced in the US in 1960-68. It stands for the American
standard code for information interchange and has 7 bits per character.
There are no support for accents, umlauts etc, but that's the yanks for
you.

There is a code chart that has four bits in the column and 3 bits in the
rows, so to access an ASCII value, you use the three bits from the row,
followed by the four from the column. 

ASCII was replaced by Unicode later on which is a single standard that is
the be all and end all. It covers most known languages and even languages
like Elvish for Christ sake.

You can't use Unicode to write anything directly because it's a standard
and not an encoding. There are a number of `flavours' of Unicode:
\begin{itemize}
    \item UCS-2: each character is exactly 2 bytes long
    \item UCS-4: each character is exactly 4 bytes long
    \item UTF-8/16: This is variable length encoding
    \item UTF-7: "e-mail safe" format (only uses bits per byte)
\end{itemize}

A file is in \textbf{text format} if it contains text in some form of
encoding. For example, the format that I'm writing in at the moment is
UTF-8. A file is in binary format if it can contain arbitrary bytes. Some
of these may not print right if you open it. 

Binary files can be really hard to display/e-mail. We can encode these
again before we email them. This is crucial for cryptographic data such as
keys, ciphertexts and signatures which are binary data. The encoding is
nothing to do with secrecy or security -- it's just so that you can type,
send, and work with these objects.

\subsection{Binary and Hex}%
\label{sub:Binary}

There are different bases that you can work with: hex, binary, and
decimal. You already know this though. Hex encoding is really common
because each byte can be encoded into exactly two characters.
Unfortunately, the encoded data is twice as long as the original. An
example of this is colours. Since computers usually encode colours as 3
bytes, each representing the red, green, and blue components, it makes
sense to make each of these using 6 hexadecimal characters. 

\subsection{Base64}%
\label{sub:Base64}
We're not talking about base 64 encryption, because it's not encryption.
It's encoding. Base64 splits data into 3-byte blocks and then encodes each
block as 4 characters. This means that the encoded data is only a third
longer than the original, rather than twice the length as with hex
encoding. If we take an example of the string "hi!":
\begin{align*}
    h = &0111 1000 \\
    i = &0110 1001 \\
    ! = &0010 0001
\end{align*}

It then takes this and splits it up into 3:
\begin{align*}
    0110 10|00 0110|1001 00|10 0001 \\
    011010 000110 100100 100001
\end{align*}

After this, it converts these into letters using a lookup table that I'm
not prepared to write out here. Basically, from this sequence of bits, we
end up with "hi!" = "aGkh". Dope, right? It turns any 3-byte sequence into
4 printable characters, therefore turning binary data into a text format. 

\section{Cryptography}%
\label{sec:Cryptography}

\subsection{Block ciphers}%
\label{sub:block}

A block cipher such as AES works by having a message, a key, and an
encryption function. The message can be of any length, and the ciphertext
should be about the same size as the message. It's a bad idea to have
something that produces a pattern, because this means that the code can be
decrypted very easily. Using a block cipher that returns the same message
when used for the same message is called ECB, which stands for electronic
code book.

The subtleties of what makes a good encryption function is beyond this
unit, but the point is that we can prove that if the block cipher kinda
looks random on single blocks, then a good mode of encryption looks random
as well. Good modes include another `random' block at the start, known as
the initialisation vector, so that you don't get the same thing if you
encrypt the same message twice. Did you get that? If you encrypt the same
thing twice, you don't get the same message! 

An example of this is CBC, which is cipher block chaining. CBC mode is
cool because it has a chain of ciphers (kind of like DES). It's really
important that \textbf{patterns} and \textbf{repetition} are be hidden.
Also, that you don't learn anything about the contents of a message, other
than it's length.

\subsection{MACs}%
\label{sub:MACs}
In a passive attack, the adversary only observes the data. It's not
interfered with at all. If the attack is \textit{active} then data can be
inserted, deleted, or meddled with. Even the delivery destination of the
data could be compromised. A message authentication code, or MAC, is an
algorithm that generates a \textit{tag} that is based on the message and a
shared secret key. The tag is sent with the message, and the recipient
confirms this by running the same algorithm with the same key and
comparing the value. If it is the same, then the recipient is assured that
the message is the same as the one originally sent. If there is a mismatch
in any way then this means that the message that was received has changed
in some way to the one that was sent in the first place. A MAC only does
what it's told to do, and since they're not required to preserve
confidentiality, sending a MAC of the plaintext message in the clear
actually compromises the confidentiality provided by the encryption
protocol. It makes much more sense to encrypt the message and then MAC it,
which ensures the integrity of the ciphertext along with the
confidentiality of the plaintext, or the other option is to MAC, and then
encrypt. Most cryptographers recommend encrypt and then MAC.

Checksums are not MACs. They're just used for detecting errors in stored
or transmitted data.

\subsection{Key management}%
\label{sub:Key management}

Symmetric cryptography provides the facility to exchange encrypted
documents between people who share the same key. This raises the question,
how do they even get those keys to start with? For $n$ people to be able
to communicate in pairs requires $n^2$ keys. They need to be agreed over
secure channels to prevent eavesdroppers from obtaining them and the just
going ham on the messages. They also need to be updated sometimes. But,
what's the next step?

\section{Public-Key Cryptography}%
\label{sec:Public}

The operation of public key cryptography is kind of like a padlock. You
can click it, and it's shut. But, when it's shut you need a key. This is
really useful for being able to communicate with a lot of people, since
they can all send you encrypted messages, but no one can read them except
from you.

It all relies on two things: A public key, and a secret, private key. The
public key must be made available to the public y making it available in
some directory. Then, anyone who wants to send a message to you encrypts
it under the public key, but it can only be decrypted by the secret key.

For this to even work, you need both of the keys (which are long and
random numbers) to be mathematically linked in some way. Normally, the
public key is made by some one-way function from the private key, and is
easily computed by the owner of the private key, while the inverse
transformation is practically impossible.

A public key encryption has three algorithms:
\begin{enumerate}
    \item \textbf{K}: the key generation algorithm. This returns the
        private and public keypair.
    \item \textbf{E}: the encryption algorithm, which encrypts the message
        with the public key, and returns the ciphertext
    \item \textbf{D}: the decryption algorithm that takes the ciphertext
        and the secret key as the input, and returns the original message
        in plaintext.
\end{enumerate}

Public key encryption by itself offers literally no authenticity. Anyone
at all can send you encrypted messages.

We're going to look at signing messages now. There can be such a thing as
a Public-Key Infrastructure (or PKI), which is a framework for generating,
managing, and distributing keypairs and digital certificates (proof of key
ownership). This means that if someone trusts a person, then they can add
their signature to the key.

\subsection{Digital signatures}%
\label{sub:signatures}

A signature is equivalent to a handwritten signature. It binds someone's
identity to a piece of information. With a digital signature verification
algorithm, one verifies that a digital signature is authentic. A digital
signature scheme consists of a key generation algorithm, a signature
creation algorithm, and an associated verification algorithm. There are
different kinds of signature schemes. Digital signatures with appendix
require the original message as an input to the verification algorithm.
Digital signatures with message recovery do not require the original
message as an input to the verification algorithm. In this case, the
original message is just recovered from the message itself.

The digital signature standard (or DSS) is a standardised signature scheme
with appendix. The DSS describes an algorithm called DSA, which is based
on the mathematically difficult problem of computing discrete logarithms
in certain groups it also includes some variant of the same algorithm on
elliptic curves, the ECDSA. In order to sign a message, the owner first
must compute the hash value of the message. This hash value and the
private key are then subjected to the signing algorithm. The output of
that algorithm is the signature of the input message. Anyone who wishes to
verify the signature can simple take the message, the public key and the
signature and complete the verification algorithm with these inputs. 

\section{Malware}%
\label{sec:Malware}
Malware is a word that means malicious software. It refers to files that
run on a machine that have intentions of the not-so-nice kind. They can do
things like:
\begin{itemize}
    \item Collect information about the user.
    \item Encrypt data on the machine with the aim to get the user to pay
        for decryption (also known as ransomware)
    \item Linking the user up to a paid service (that is useless)
    \item Installing a piece of software that will be activated in the
        future.
\end{itemize}

The problems are getting worse, and they're already pretty bad. Before the
internet, systems could get corrupted by malicious floppy disk freebies,
but because everyone uses the internet, it's way easier to spread vicious
programs.

There are a lot of types of malware:
\begin{itemize}
    \item Some types, such as viruses and worms, spread by replicating.
    \item Some are concealed, such as trojans and logic bombs.
    \item Types such as adware, spyware, pornware and ransomware are
        financially motivated.
    \item Some, such as rootkits, are designed to gain control of the
        machine, and then like them up to a botnet.
    \item Viruses are pieces of code that rely on some other piece of code
        to propagate. They have three components:
        \begin{itemize}
            \item An infection mechanism
            \item A trigger, which is some condition such as time, under
                which it will begin to execute
            \item A payload, which is the malicious code
        \end{itemize}
    \item A worm is a \textit{self-replicating} virus. This means that
            it makes copies of itself, without relying on a host to
            spread. It achieves this by exploiting weaknesses of networks,
            networked services and operating systems, like:
    \begin{itemize}
        \item Using a (command line) email client to email itself to other
            users.
        \item Using (unprotected) user credentials to log into other
            systems remotely, to copy itself over and execute
    \end{itemize}
    \item Logic bombs are pieces of code that are inserted into a system.
        It functions as a trigger when specified conditions are met, such
        as the commencement of a process to delete particular files from
        an employer's system if their employment is terminated.
    \item A trojan is a piece of code that has secretly been inserted with
        harmful intent. They're often activated by logic bombs. Most forms
        of adware etc. are trojans.
\end{itemize}

Malware is everywhere. Main attack vectors are USB devices sometimes, but
more often from adware and downloads and emails.

It's pretty hard to detect malware because the really effective ones are
\textit{polymorphic} meaning that they change their program code such that
the functionality remains, but their code is different, kind of like a
real virus. They try to use more \textbf{obfuscation techniques} than
benign programs in an attempt to hide memory patterns which risk
identifying them. It attempts to access sensitive files, change
permissions, or connect to the network are also warning signs to look out
for. 

\section{Humans are dumb lol}%
\label{sec:dumb-humans}

Whenever we discuss and analyse security risks in cyberspace, you have to
consider how humans are going to interact with the systems that we are
analysing. For example, we have trouble remembering long and random
strings. We're really social, so the responses to computers or technology
are based on what they learn from interactions with other people. This is
just the way society be. People are generally quite kind and considerate,
so it's pretty easy to exploit that and use their kindness to make them do
stupid things like give away their passwords. Unfortunately, anonymity
(perceived though it may be) encourages dishonesty. We see this through
things like trolling and stuff. It's probably from the feeling of power
through anonymity that raises self-esteem. It could also be because
certain online communities attract egocentric personality types.
Psychologists even describe deindividuation in extreme cases. This is
where inner restraints are lost when individuals are not seen. Basically,
trolls can do whatever they like, because they don't see beyond the
screen. 

It's also easier to lie 
\end{document}
