%%=====================================================================================
%%
%%       Filename:  database-n-cloud.tex
%%
%%    Description:  Notes for database and cloud written up nicely
%%
%%        Version:  1.0
%%        Created:  06/04/19
%%       Revision:  none
%%
%%         Author:  Josh Felmeden (), nk18044@bristol.ac.uk
%%   Organization:  
%%      Copyright:  Copyright (c) 2019, Josh Felmeden
%%
%%          Notes:  
%%
%%=====================================================================================

% Preamble
\documentclass[11pt,a4paper,titlepage,dvipsnames,cmyk]{scrartcl}
\usepackage[english]{babel}
\typearea{12}

% Set indentation and line skip for paragraph
\setlength{\parindent}{0em}
\setlength{\parskip}{1em}
\usepackage[margin=2cm]{geometry}
\addtolength{\textheight}{-1in}
\setlength{\headsep}{.5in}

% Headers setup
\usepackage{fancyhdr}
\pagestyle{fancy}
\lhead{Databases and the Cloud: The notes}
\rhead{Josh Felmeden}
\usepackage{hyperref} 
\usepackage{mathtools} 


% Listings
\usepackage[]{listings,xcolor} 
\lstset
{
    breaklines=true,
    tabsize=3,
    showstringspaces=false
}


\lstdefinestyle{Common}
{
    extendedchars=\true,
    language={HTML},
    frame=single,
    %===========================================================
    framesep=3pt,%expand outward.
    framerule=1.4pt,%expand outward.
    xleftmargin=3.4pt,%make the frame fits in the text area. 
    xrightmargin=3.4pt,%make the frame fits in the text area.
    %=========================================================== 
    rulecolor=\color{Red}
}

\lstdefinestyle{A}
{
    style=Common,
    backgroundcolor=\color{Yellow!10},
    basicstyle=\small\color{Black}\ttfamily,
    keywordstyle=\color{Orange},
    identifierstyle=\color{Cyan},
    stringstyle=\color{Red},
    commentstyle=\color{Green}
}

\lstdefinestyle{B}
{
    style=Common,
    backgroundcolor=\color{Black},
    basicstyle=\scriptsize\color{White}\ttfamily,
    keywordstyle=\color{Orange},
    identifierstyle=\color{Cyan},
    stringstyle=\color{Red},
    commentstyle=\color{Green}
}

\usepackage[]{amsmath} 
\usepackage[]{booktabs} 
\usepackage[symbol]{footmisc} 
\renewcommand{\thefootnote}{\fnsymbol{footnote}}

% Title
\title{Databases and the Cloud: The Notes}
\date{2018\\ December}
\author{Josh Felmeden}

% Start document
\begin{document}
\pagenumbering{roman}
\maketitle

\tableofcontents
\newpage
\pagenumbering{arabic}

\section{The Internet}%
\label{sec:The Internet}
End systems are connected via the \textbf{communication links} that
consist of the different types of physical media. Usually, the end systems
are not directly attached by a single link, but rather they are attached
through a router.

There are two kinds of host: \textit{clients} and \textit{servers}. A
program or machine that responds to request and others is called a
\textbf{server} while a program or machine that sends the requests to the
server is called a \textbf{client}.

The internet is made possible by the development, testing, and
implementation of the \textit{internet standards}. They are developed by
the Internet Engineering Task Force (or the IETF). Their documents are
known as RFCs (request for comments). There are a number of protocols,
such as TCP, IP, HTTP, and SMTP (this one is used for emails). There are
more than 2000 RFCs.

\subsection{Protocols}%
\label{sub:Protocols}
A \textbf{protocol} is a set of rules that govern the communication to
ensure a standard of communication. It also consists of messages sent and
actions taken in response to replies or other such events.

A simple protocol could be where one machine sends a message (called a
\textit{request}) and another machine replies with a response. This can
then be repeated.

\subsection{Internet Layers}%
\label{sub:internet-layers}
\begin{itemize}
    \item HTTP
    \begin{itemize}
        \item Makes request
        \item Reads and handles the response
    \end{itemize}
    \item TCP
        \begin{itemize}
            \item Breaks data up into packets
            \item Puts the packets back in order and reassembles messages
        \end{itemize}
    \item IP
        \begin{itemize}
            \item Attaches to and from addresses to each packet
            \item Reads and groups packets based on the address
        \end{itemize}
    \item Physical internet
    \begin{itemize}
        \item Send bits to local routers
        \item Receives bits and assembles into packets
    \end{itemize}
\end{itemize}

\subsection{HTTP: Hyper text transfer protocol}%
\label{sub:http}
What's the difference between the web and the internet? Well, the internet
is the computer network itself (or the whole infrastructure): while the
web (or the world wide web) is an application that runs on that
infrastructure.

It's probably the most common application protocol that there is on the
web (but there are others like video streaming and FTP and the like).
Right now, there's a version 2.0, but we'll be focussing on version 1.1
here.

\subsection{Crud}%
\label{sub:Crud}
CRUD is an acronym for the basic operations that can be carried out on
data.

\begin{itemize}
    \item Create
    \begin{itemize}
        \item The create interaction creates a new resource in a server
            assigned location. The create interaction is performed by a
            HTTP POST method.
    \end{itemize}
    \item Read
    \begin{itemize}
        \item The read interaction accesses the current contents of a
            resource. The interaction is performed by a HTTP GET method
    \end{itemize}
    \item Update
    \begin{itemize}
        \item The update interaction makes a whole new version for an
            existing resource (or makes a new one if there isn't one)
    \end{itemize}
    \item Delete
    \begin{itemize}
        \item The delete interaction deletes an existing resource
    \end{itemize}
\end{itemize}

\subsection{Structure}%
\label{sub:Structure}
HTTP is \textit{line-based} and each line ends with a \textbf{carriage
return line feed} (CR LF). In it, there is a header and a method.

\subsection{Status codes}%
\label{sub:Status codes}
There are some cases where an interaction does not go well. The response
from a server can be a number, and the first digit informs you of the
nature of the error.

\subsection{URLs}%
\label{sub:URLs}
The internet needs to have addresses. It needs to know the addresses of
both the client and the server. The URL (\textbf{uniform resource
locator}) tells you where some resource is. A resource is an
\textit{address}.

\section{Developing web pages}%
\label{sec:web-pages}

\subsection{Markup}%
\label{sub:Markup}
Historically, marking up a paper manuscript was done by editors to show
authors how to revise their manuscripts. The markup was done in
\textit{blue pen} to make it distinguishable from the manuscript text.

In electronic documents, \textbf{tags} are used to make the markup
distinguishable from the content. A markup language is used to annotate a
document.

\subsection{HTML}%
\label{sub:HTML}
HTML (or hypertext markup language) consists of a fixed set of
\textit{tags} that describe how information should be displayed. For
example:
\begin{lstlisting}[style=B]
<p> This is some text </p>
<h1> This is some header </h1>
\end{lstlisting}

The Browsers do not display the HTML tags, but they use them to render the
content of the page.

HTML5 is different from HTML because it's simpler, but also \textbf{semantic}
(which means that some of the tags describe what the data means as well
has how it should be displayed). It also has some more features.

Example HTML5:
\begin{lstlisting}[style=B]
<!DOCTYPE html>
<html lang="en">
    <head>
        <meta charset="utf-8" />
        <title>My title</title>
    </head>
    <body>
        content
    </body>
</html>
\end{lstlisting}

Tags have to be nested too.

A block-level element always starts on a new line and takes up the full
width available. Conversely, an inline element doesn't start on a new line
and it only takes up as much width as is necessary.

The \lstinline|<div>| tag is a block level tag that has no specific
meaning. This is OK to use for layout purposes, but you should not use it
as a replacement for something that should be a semantic tag. Because the
semantic tags are mostly a new addition to HTML5, older frameworks used
\lstinline|<div>| all over the place to structure the pages.

\subsubsection{Attributes}%
\label{ssub:Attributes}
In this example:
\begin{lstlisting}[style=B]
<p id="today">
    28 September
</p>
<p class="info">
    lecture 2
</p>
<p class="info">
    QB 0.18
</p>
\end{lstlisting}

\subsubsection{Links}%
\label{ssub:Links}
Almost anything can go inside a \lstinline|<a>| tag: text, images, other
HTML elements. The href could be a full URL, or it can be relative to the
current page.

The main issue with HTML is that you need to structure your web pages
really carefully because it's going to be viewed on all kinds of devices
and browsers.

Here are some basic rules:
\begin{enumerate}
    \item Use lower case element names
    \item Close all your elements (you don't need to close them in HTML5
        but do it anyway)
    \item In HTML5, it's optional to close the empty statements, but do it
        anyway.
    \item HTML5 allows the mixing of uppercase and lowercase names, but
        just use lowercase because it looks nicer and it's easier to
        write.
    \item HTML5 allows attribute values without quotes but again, it's bad
        because it looks ugly
    \item ALWAYS add the \lstinline|alt| attribute to images, because if,
        for some reason, the image can't be displayed, you need some
        alternate text to display. It's also used for people using screen
        readers.
    \item In HTML5, the \lstinline|html| and \lstinline|body| tag can be
        omitted, but, again, it's \textbf{bad}. It can crash some XML
        software.
    \item To ensure that everything is interpreted and has correct search
        engine indexing, the language AND the character encoding should be
        defined as early as possible.
    \item Don't use absolute pixel width measurements
\end{enumerate}

\subsection{Forms}%
\label{sub:Forms}
The form tag is used for things like buttons, text boxes, etc. It has
input types of things like:
\begin{itemize}
    \item Button
    \item Month
    \item Number
    \item Text
    \item Password
    \item Color
    \item Date
    \item ...
\end{itemize}

The \textbf{action} attribute defines the action to be performed when the
form is submitted. Normally, the data from the form is sent to a web page
on the server when the user clicks on the submit button. For example:

\begin{lstlisting}[style=B ]
<form method="post"
action="/action_page.php">
</form>
\end{lstlisting}

In this example, the data is sent to a page on the server called
\lstinline|"/action_page.php"|. This page contains a script that will
handle the form data such as storing it in a database.

There are two (2) methods to send form data, \textbf{GET} and
\textbf{POST}. In HTML5, browser forms support them both. GET places form
data in the URL parameters by default (GET/search?query=pancakes), while
POST sends the data in the HTTP request body. There are fewer limitations
and it's more secure because the data is not visible in the URL.

\subsubsection{Validation in forms}%
\label{ssub:form-validation}
If you use \lstinline|type="number"|, then it won't let you type in
letters. It you use \lstinline|required|, then the browser won't let you
submit if the field is empty.

Place holder is text that can be displayed while the field is empty. It's
NOT a label.

\subsubsection{Buttons}%
\label{ssub:Buttons}
Buttons can have these types:
\begin{itemize}
    \item Submit (default)
    \item Reset: reset all form fields
    \item Button: do nothing by default (use this if you're using JS).
\end{itemize}

\subsection{CSS}%
\label{sub:CSS}
Use HTML for the structure, and then CSS for the styling. This includes
layout, appearance, and some behaviours. You can customise ANYTHING. If
you want emphasised words to be underlined, then by golly 




\end{document}
